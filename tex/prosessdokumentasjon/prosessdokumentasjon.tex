\chapter{Prosessdokumentasjon}
%\marginpar{
%Test for sitering av kilder.\cite{forelesning:tulpesh}\cite{book:utforming}\cite{book:desintsystems}
%}
\lettrine[lines=2]{F}{} ølgende kapittel beskriver hvordan hele utviklingsprosessen av prototypene har foregått og som illustrerer hele prosessen fra idé til mockup til hi-fi prototype. Kapittelet består av mange bilder for å på enklest mulig måte illustrere prosessen. For eventuell beskrivelse av funksjonalitet av de modulene som er synlige på bildene henvises det til appendiks \ref{app:funksjonalitet}.




\section{Første utkast} \label{sec:utkast}
%\marginpar{Viktige prinsipper:
%feedback, constraints (bruker får ikke gjøre feil), affordances
%}
%\marginpar{Husk at vi borde ta med en navigasjonkart}
%\marginpar{Støtteord:
%Usability (brukbarhet): konsistens, brukerkontroll, passende presentasjon.
%}
Etter at idéen om hva vi skulle jobbe med var klar tok det ikke lang tid før vi ble enige om hvordan vi skal sette sammen vårt forslag til en helhet. Hele gruppen var tydelig på at vi i stor grad ønsket å benytte oss av en løsning som følger \textit{Gestalt}-prinsipper som en underliggende tommelfingerregler for en god design og layout.
Systemet skulle kunne administreres via nettleser som gjør det mulig å bruke systemet fra hvilken som helst maskin og med noen tilpasninger kan også gjøres mulig for å jobbe fra nettbrett eller telefon.
På det tidspunktet handlet diskusjonen om hvordan vi skulle fordele de forskjellige modulene og funksjonalitetene på så få hovedområder som mulig, der hvert av de områdene blir et element i hovedmenyen. 
I det første utkastet bestod disse av totalt fem forskjelilge seksjoner som presenteres i tabell \ref{tab:sidekart}.
%\begin{center}
%\texttt{Editorer | Meldinger | Resurser | System | Servere}
%\end{center}
\begin{table}[h]
\caption[Sidekart]{Enkel sidekart over det første utkastet. Tabellhode representerer hver enkel menykategori. I rad to presenteres forslag til hver menykategori.}
\label{tab:sidekart}
\newcommand{\paddA}{0.5ex}
\newcommand{\paddB}{0.2ex}
\renewcommand{\familydefault}{\ttdefault}\normalfont
\begin{tabularx}{\textwidth}{|*5{>{\centering\arraybackslash}X|}@{}|}
\hline
\vspace*{\paddA} Editorer & \vspace*{\paddA} Resurser & \vspace*{\paddA} Meldinger & \vspace*{\paddA} System & \vspace*{\paddA} Servere \\[2ex] 
\hline
\vspace*{\paddB} Java		&	\vspace*{\paddB} HTML Val			& 	\vspace*{\paddB} Ny 	& 	\vspace*{\paddB} Apache 	& \vspace*{\paddB} Brannmur \\
HTML\& PHP		& 	CSS Val		 	& 		Les		& 	MySQL 	& Brukerg \\
MySQL	& 	W3Schools 		& 				& 	SSH 		& Logg \\
C/C++/C\# 		& 	DropBox 			& 				& 			& Filbehnadling \\
 		& 	Terminal 		& 				& 			& Systeminfo \vspace*{0.2ex} \\

\hline
\end{tabularx} 
\end{table}

Idéen ble deretter tegnet i form av en iterativ prosess i mellom gruppens medlemmer. Vi ble til slutt enige om et sluttresultat for idéen som vi godkjente for å bygge videre på. Det siste skissen presenteres i figur \ref{fig:foersteutkast}.
\begin{figure}[h]
%bruk \begin{figure}[ht] dersom figuren ikke skal flyte
\includegraphics[width=\textwidth,height=\textheight,keepaspectratio]{./img/prosessdokumentasjon/foersteutkast/foerste.jpg}
\caption[Første utkast]{Første utkast over brukergrensesnittet.}
\label{fig:foersteutkast}
\end{figure}

\section{Low-fi prototype} \label{sec:lowfi}
\emph{Avsnittet inneholder et stort antall mockups som representerer forskjellige skjermbilder. Det er flere bilder som er plassert i teksten, og dersom noen av komponentene i bildene oppleves for små er samtlige av bildene presentert i høy oppløsning i appendiks \ref{sec:appendiksLowFi} side \pageref{sec:appendiksLowFi}.}

Det er ganske vanlig at man til en low-fi prototype bruker papp eller papir for å visualisere hvordan man kan bruke en GUI. Vi valgte å benytte oss av \href{http://balsamiq.com/products/mockups/}{\textit{Balsamiq Mockups}} som ga oss mulighet til å ikke bare visualisere hvordan brukergrensesnittet skulle se ut men også legge inn enkel funksjonalitet. Blant annet at man fikk mulighet til å klikke seg videre til neste skjermbilde direkte ved å trykke på fra menyer eller knapper i mockupen (en linkbasert mockup \cite{book:utforming}). Slik utforming gav en fingervisning over hvordan det kan bli å bruke det ferdige grensesnittet.

Dersom vi tar et rask overblikk over framsiden i figur \ref{fig:lowfi_fremside} ser vi at hver enkelt modul er gruppert etter gestalt prinsipper for proksimitet og elementer.\cite{forelesning:tulpesh}
Prototypen inkluderer eksempel på konfigurasjon av to forskjellige funksjoner.
Det er vanlig at i en prototype fokuserer man på en horisontal eller vertikal implementasjon. Der man i den horisontale versjonen implementerer få eller ingen funksjoner som går i dybden med istedet forsøker man å vise bredden av muligheter i et system eller grensesnitt. I den vertikale implementasjonen velger man insteden å begrense bredden på hva prototypen kan vise eller gjøre med insteden implementerer man en eller flere funksjoner på dybdeben.\cite{book:utforming}
I dette studium har vi valgt å avgrense disse til konfigurasjon av \textit{Apache} webserver og oppsett av brukere og grupper. 

\begin{figure}[ht]
%bruk \begin{figure}[ht] dersom figuren ikke skal flyte
\includegraphics[width=\textwidth,height=\textheight,keepaspectratio]{./img/prosessdokumentasjon/lowfi/fremside.png}
\caption[Low-fi prototype]{Fremside for EasyDev i første low-fi prototype.}
\label{fig:lowfi_fremside}
\end{figure}


I etterfølgende seksjoner beskriver vi i detalj funksjonaliteeten til to moduler. Disse er grunnelgende moduler for systemet som har følgt med siden den ursprunglige idéen. Disse moduler er et godt eksempel på hvordan vi har tentk at produktet skal fungere. Derfør ønsker vi å beskrive de mer detaljert.

\subsection{Oppsett av apache webserver}
Her tentke vi gå nærmere på å vise hvordan det er tenkt at en submodul for konfigurasjon og oppsett av serverelement skal brukes. Alle trinn som beskrives er presentert i figur \ref{fig:lowfiapache}.

Vi utgår fra at dette er intuitiv å velge modul \texttt{Servere} dersom man ønsker å konfigurere en tjener. Modulen er tilgjengelig på fremsiden med en rullmeny som viser hvilke servere som vi har mulighet til å konfigurere, figur \ref{fig:apache1}. 
\textit{Apache} webserver trenger å vite hvilke mapper som denne skal lese fra. Hvis man installerer apache på en Linux manskin kommer den som standard å lese fra mappe \texttt{/var/www/} som har kun \texttt{root} tilgang. Vi vil ikke endre rettigheter for denne mappe ettersom de vi stride mot standarden for rettigheter i systemet og utgjøre en sikkerhetsrisiko.\cite{book:unixprog}
Det som vi ønsker å gjøre isteden er å definere om instllinger i \textit{Apache} slik at serveren leser fra brukerens sin egen mappe. Dette er normalt en utfording ettersom det må gjøres i form av to til tre trinn\footnote{Antall trinn her er ganske avhegig av hvilket Linux distribusjon som brukes}:

\begin{enumerate}
\item Endre konfigurasjon i apache.conf til å istedenfor eller i tillegg lese fra brukerens hjemmemappe. Det må opprettes en spesifikk mappe f.eks. \texttt{/home/bruker/html/}. 
I noen tilfelder er det nødvendig å installere plugin \href{http://httpd.apache.org/docs/2.2/mod/mod_userdir.html}{\texttt{mod\_{}userdir}}. Dette vil gjøre det mulig å lese filene vis følgende adresse: \texttt{http://localhost/\~{}bruker/fil.html}

\item Endre rettigheter til den mappe i hjemmemappe som filene skal leses fra. Dette forutsetter at brukeren og \textit{Apache} webserver er i samme brukeregruppe. Oftest må brukeren bli lagt til \texttt{www} eller \texttt{httpd} gruppe på systemet. Det er selvfølgelig mulig å gi rettigheter for mappen til alle grupper i systemet men slik tilnærming senker sikkerheten i stor grad i et flerbrukersystem. Dette er noe som lager en del vanskeligheter for nybegynnere og der man i starten faktisk velger å fullstendig åpne rettigheter til mappen. Det gjør man ettersom man trenger som regel to brukere med forskjellige rettigheter for å nytt oppsett over rettigheter i et system. Det er dessverre noe som man ikke har på et system som man selv ikke administrerer. 

\item Systemer som benytter seg av SELinux (\href{http://en.wikipedia.org/wiki/Security-Enhanced_Linux}{\textit{Security Enhanced Linux}}) må i tillegg legges til en policy der \textit{Apache} skal få lov til å lese fra brukermapper. Dette er på grunn at SELinux er et oppsett av rettigheter som overstyrer hvanlige skriv og leserettigheter i systemet på kjerne nivå\footnote{Dette er ikke noe som vi skal beskrive i detalj og blir nevnt her kun for å illustrere problemstilligen som bruekren kan bli utsatt for.}. 
\end{enumerate}

Eksemplene over viser hvor vanskelig det kan være å sette opp Apache webserver på sin egen maskin. Det er rett og slett mulig å lage en liknende liste med oppgaver/forhinder for hver skjermbilde som vi presenterer i vår mockup. Med andre ord hvert enkelt bilde i våre mockups representerer et fleretall oppgaver som kjøres i bakgrunnen på systemet, noe som en bruker må utføre helt manuelt dersom de ikke benytter seg av EasyDev.
Dette er et primært ekesempel på område der EasyDev kommer inn med reddning. 

\textit{Apache} modulen i EasyDev er tenkt at den skal opprette alle disse konfigurasjoner for å få en fungerende server.
Dersom vi se i figur \ref{fig:apache2} ser vi at under oppsettet kan brukeren velge hvilke mapper som skal inkluderes i konfigurasjonen. Dette er mapper som webservern kommer til å ha mulighet å lese fra. 
I neste trinn (figur \ref{fig:apache3}) blir brukeren presentert i hvilke av de valgte mappene ønsker man å benytte for å aktivere <<\textit{directory browsing}>>. Det er en funksjon som gjør det mulig å se innholdet i mappe se tjenes av serveren direkte i nettleseren. Dette er normalt en funksjon som ikke brukes på en <<live>> server men er meget brukbart ved både utvikling og testing.
 
Figur \ref{fig:apache4} viser hvilke typer av innstiksplugins som kan velges i tillegg til \textit{Apache} installasjonen. Disse kan for eksempel være plugins som PHP eller MySQL eller andre. Hensikten er at man kan samle alle slike tillegg på en plass slik at brukeren kan også velge disse i et seinere tilfelde.
Siste eksempel (figur \ref{fig:apache5}) viser mulighet for konfigurasjon av noen av mer avanserte tillegsmoduler. Disse kan f.eks. være støtte for flere vituelle servere av Apache noe som gir bedre skalerbarhet ved utvikling av flere prosjekter parallellt. 
% med flagg p setter vi hele denne figuren på en egen side
\begin{figure}[p]
        \centering
        \begin{subfigure}[b]{0.48\textwidth}
                \includegraphics[width=\textwidth]
                {./img/prosessdokumentasjon/lowfi/apache1.png}
                \caption{Fremside -> Sett opp webserver}
                \label{fig:apache1}
        \end{subfigure}
        \begin{subfigure}[b]{0.48\textwidth}
                \includegraphics[width=\textwidth]
                {./img/prosessdokumentasjon/lowfi/apache2.png}
                \caption{Valg av mapper}
                \label{fig:apache2}
        \end{subfigure}
       
        \vspace{0.6cm}
        \begin{subfigure}[b]{0.48\textwidth}
                \includegraphics[width=\textwidth]
                {./img/prosessdokumentasjon/lowfi/apache3.png}
                \caption{Aktivere visning av innhold i mapper}
                \label{fig:apache3}
        \end{subfigure}
        %\hspace{0.05cm}
        \begin{subfigure}[b]{0.48\textwidth}
                \includegraphics[width=\textwidth]
                {./img/prosessdokumentasjon/lowfi/apache4.png}
                \caption{Valg av moduler}
                \label{fig:apache4}
        \end{subfigure}
        
        \vspace{0.6cm}
        \begin{subfigure}[b]{0.48\textwidth}
                \includegraphics[width=\textwidth]
                {./img/prosessdokumentasjon/lowfi/apache5.png}
                \caption{Avanserte Apache moduler}
                \label{fig:apache5}
        \end{subfigure}
        \vspace{0.1cm}
        \caption[Konfigurasjon av Apache webserver]{Eksempel på konfigurasjon av Apache webserver.}\label{fig:lowfiapache}
\end{figure}

\pagebreak
\subsection{Oppsett av brukere og grupper}
I følgende modul har vi mulighet til å sette både brukergrupper, brukere osamt rettigheter for disse. Modulen er tenkt å brukes da man tenker å benytte systemet for gruppearbeid slik at flere bruekere kan dele på samme resurser. De eksempler som beskrives her kan betraktes i figur \ref{fig:lowfibrukere}. 
% med flagg p setter vi hele denne figuren på en egen side
\begin{figure}[h]
        \centering
        %\hspace*{-0.02\textwidth}
        \begin{subfigure}[b]{0.48\textwidth}
                \includegraphics[width=\textwidth]
                {./img/prosessdokumentasjon/lowfi/b1.png}
                \caption{Fremside -> Velg brukeroppsett}
                \label{fig:brukere1}
        \end{subfigure}
        \begin{subfigure}[b]{0.48\textwidth}
                \includegraphics[width=\textwidth]
                {./img/prosessdokumentasjon/lowfi/b2.png}
                \caption{Grupper}
                \label{fig:brukere2}
        \end{subfigure}
        
        \vspace*{0.4cm}
       
        \begin{subfigure}[b]{0.48\textwidth}
                \includegraphics[width=\textwidth]
                {./img/prosessdokumentasjon/lowfi/b3.png}
                \caption{Brukere}
                \label{fig:brukere3}
        \end{subfigure}
        %\hspace{0.02cm}
        \begin{subfigure}[b]{0.48\textwidth}
                \includegraphics[width=\textwidth]
                {./img/prosessdokumentasjon/lowfi/b4.png}
                \caption{Moduler og resurser for gruppe}
                \label{fig:brukere4}
        \end{subfigure}
        %\vspace{0.1cm}
        \caption[Konfigurasjon av brukere og grupper]{Eksempel på konfigurasjon av brukere og grupper.}\label{fig:lowfibrukere}
\end{figure}
Vi begynner på samme måte som i forrige eksempel å starter på fremsiden (figur \ref{fig:brukere1}). Istedenfor å velge modul servere går vi til modul <<System>> som innholder flere undermoduler som kan brukes til oppsett og konfigurasjon av selve virtuelle maskinen og brukergrensesnittet. 
Vi kan derette (figur \ref{fig:brukere2}) sette opp en ny gruppe, slette gruppe eller gå videre med å legge til brukere til en gruppe. Under kontrollene ser vi også hvilke resurser som alerede er tildelt den gruppen og evntuelt kan vi fjerne disse rettigheter gjennom å sjekke en eller flere av alternativene.
Dersom vi ønsker å legge til nye bruekre i systemet vårt kan vi gjøre detet i neste skjermbilde (figur \ref{fig:brukere3}) og på en enkel måte gå videre og legge til eller slette brukere.
Hvis vi ønsker kan vi gå videre til neste skjermbilde å legge til nye resurser som skal være tilgjengelige for spesifikk gruppe. Dette kan f.eks. dreie seg om en database, mappe alle sånn som tilgang til en konto på FTP server som kjører på systemet. 

\section{Hi-fi prototype}
\emph{Følgende avsnitt innholder ikke alle figurer over ferdig hi-fi prototype. Det er et flertall skjermbilder som må vises i full tekstbredde for at de skal være lesebare. Samtlige hi-fi mockups er derfor flyttet til appendiks \ref{sec:appendiksHiFi} side \pageref{sec:appendiksHiFi}.}

Hi-fi prototypen er en viderutvikling av tidligere prototype som er gjort med hjelp av samme verktøy som skal brukes til utvikling av ferdig produkt. 
Ved fremtaging av hi-fi prototype ble det brukt  html, css, javascript og php for å sy sammen funksjonalitet og få den til å fungere som en nettside. I figur \ref{fig:hifi_fremside} ser hvordan fremsiden blir presentert. Dette kan gjerne sammenliknes med den tidligere mockupen figur \ref{fig:lowfi_fremside} side \pageref{fig:lowfi_fremside} med hensikt å synnliggjøre utviklingsprosessen fra mockup til prototype.
\begin{figure}[ht]
\includegraphics[width=\textwidth,height=\textheight,keepaspectratio]{./img/prosessdokumentasjon/hifi/fremside.png}
\caption[Hi-fi prototype]{Fremside for EasyDev som hi-fi prototype.}
\label{fig:hifi_fremside}
\end{figure}

Det er gjort litt annerledes tilnærming til hvordan dialogene til brukeren blir presentert i hi-fi prototypen. Nærmere eksempel på dette presenteres i figur \ref{fig:hifi_brukerdialog}. 
Dialogen blir presentert lengre ned i forhold til den globale menye, dette gjør at store deler av tomt område på fremsiden blir isteden brukt for visning av dialoger. 
Forandringer er gjort siden den første mockupen med hensikt for å ikke dekke menykomponenter. 
Forandringen vil gjøre det mulig for å velge en annen modul direkte uten å behøve avslutte den aktive modulen (forutsatt at alle andringer som den aktive moduler gjør er blitt lagret).
Med andre ord kan vi lukke en dialogboks for at direkte starte en annen.

\begin{figure}[ht]
\centering
\includegraphics[width=0.8\textwidth,keepaspectratio,trim = {12cm 6cm 12cm 8cm}, clip]{./img/prosessdokumentasjon/hifi/a2.png}
\caption[Hi-fi brukerdialog]{Eksempel over fremstilling av brukerdialog. Utklipet viser eksempel på ny design og hvor dialogen kan bli plassert i forhold til menyen.}
\label{fig:hifi_brukerdialog}
\end{figure}

\section{Kriterier og avgjøresler}
Følgende avsnitt beskriver hvilke kriterier og valg som ble tatt under reisen. Vi skal fokusere på valg kriterier og valg fra et MMI perspektiv men det er også viktig å opplyse av andre kriterier som mer teknisk karaktær.

\subsection{Virtuell maskin}
System skal kjøre i en vituell maskin ettersom det er enekelt å både opprette slette eller gjenoprette systemet. Da en ny bruker skal starte en maskin trenger den kun bli kopiert fra en avbildning og satt opp med riktig brukernavn og passord. Dersom brukeren ødelegger noe i maskinen blir det muligå eksportere konfigurasjonsfiler for valgte moduler og oprette en ny maskin med sammen med eksisterende konfigurasjonfiler f.eks. \textit{Apache}.

\subsection{Webbasert}
Ettersom grensesnittet er webbasert blir dt tilgjengelig fra hvilken datamaskin som helst uavhengig operativsystem eller nettleser. Systemet vil få en god skalering da  det blir enkelt å sette inn eller fjerne moduler. Det vil også være relativ liten terskel for studenter å utvikle egene moduler dersom man sysnes at noen vesntlige moduler mangler i systemet. 

\subsection{Bruk av visuelle elementer}
Som tidligere beskrevet i avsnitt \ref{sec:utkast} og \ref{sec:lowfi} har vi valgt å strikt benytte oss av gestalt prinsipper ved plassering av alle visuelle kompoenter. 
Hensikten med dette er at når brukeren for første gang besøker fremsiden til systemet, skal det være fulstendig intuitivt, uten å behøve tenkte på hvordan forskejllige kontroller er plasert. 

Et godt eksempel på dette er plasering av alle knapper på fremsiden (se figur \ref{fig:hifi_fremside} side \pageref{fig:hifi_fremside}). 
Disse har vi valgt å legge horisantalt med samme avstand fra hverandre. 
Slik plassering danner tilhørighet, slik at man veit hvilke komponenter som henger sammen (\textit{proximity, element connectedness, similarity}). I tilleg forsterkes dette gjennom at vi bruker samme farger på alle de komponentene.\cite{forelesning:tulpesh}
Samme prinsipper gjelder også for de undermenyer som åpnes dersom man klikker på noen av de konmponentene. Man blir presentert med en liste av muligheter gruppert etter tilhørighet til sin funksjon. 

Toppanelen er symetrisk distribuert for å danne en følelse av balanse og sentral perspektiv i layouten. Vi har plasert EasyDev logo i midten og på begge sider av de to komponenter som er kalender og klokke til venstre samt brukerikone og hurtivalg til instilligenr til høyre om logoen. Dette danne en tydelig fordelig og utnyttelse av overflate som ikke kan grupperes sammen etter funksjonstilhørighet med hovedkomponentene på siden.

Nederste rad har vi i prototypen valgt å bruke til elementer som har med status av systemet å gjøre, noe som foreløpig ble forkastet i hi-fi prototype. 
For tilfeldet er det tenkt at man direkte etter pålogging kan får en oversikt over cpu og ram bruk, samt tilgjengelig lagringsutrymme og kjørende prosesser (se mockup figur \ref{fig:lowfi_fremside} side \pageref{fig:lowfi_fremside}). 
Dette er noe som kan være ønskeverdt å se for noen av brukerne med det er foreløpig usikkert dersom slike funksjoner skal være aktivert for standardbrukere og derfør er fireløpig ikke implementer i hi-fi prototype. Det finnes også forslag at utrymmet kan reserveres til andre foremål som brukeren kan selv velge hvilke typer av <<widgets>> som kan benyttes. Dette er tillegsfunksjonalitet som kan bli i stor grad forendret fra utviklig av prototype til alfa versjon.

\subsection{Bruk av farger}
Vi har konsekvent valgt å avgrense oss til grunpaletten i \textit{FlatUI} grensesnittet. Alle farger som vi kan bruke i løsningen presenters i form av en matrise i figur \ref{fig:farger}. Dette er $4\times 5$ matrise som kun har tilgang til tjue farger men gir oss ganske store muligheter. 
\begin{figure}[h]
\begin{tabularx}{\textwidth}{*6{>{\centering\arraybackslash}X}@{}}

\cellcolor{Turquoise} \emph{Turquoise} & \cellcolor{Emerald} \textit{Emerald} & \cellcolor{PeterRiver} \emph{Peter River} & \cellcolor{Amethyst} \emph{Amethyst} & \cellcolor{WetAsphalt} \emph{Wet Asphalt} \\[5ex] 

\cellcolor{GreenSea} \emph{Green Sea} & \cellcolor{Nephritis} \textit{Nephritis} & \cellcolor{BelizeHole} \emph{Belize Hole} & \cellcolor{Wisteria} \emph{Wisteria} & \cellcolor{MidnightBlue} \emph{Midnight Blue} \\[5ex] 

\cellcolor{SunFlower} \emph{Sun Flower} & \cellcolor{Carrot} \textit{Carrot} & \cellcolor{Alizarin} \emph{Alizarin} & \cellcolor{Clouds} \emph{Clouds} & \cellcolor{Concrete} \emph{Concrete} \\[5ex] 
 
\cellcolor{Orange} \emph{Orange} & \cellcolor{Pumpkin} \textit{Pumpkin} & \cellcolor{Pomegranate} \emph{Pomegranate} & \cellcolor{Silver} \emph{Silver} & \cellcolor{Asbestos} \emph{Asbestos} \\[5ex] 

\end{tabularx} 
\caption[Fargekombinasjoner]{Fargekombinasjoner som brukes i hi-fi prototypen}
\label{fig:farger}
\end{figure}
Dersom vi ser på to øverste og to nederste rader ser vi at disse er bare forskejllige nyanser av samme farge. Disse har vi valgt å bruke som utheving eller for å danne lav kontrast mellom feks. tabellrader. 

Dersom vi trenger mer kontrast for å varsle brukeren om noe kan vi ta i bruk to \textbf{analoge} farger. Disse er plassert over hverandre i rader 1-3 eller 2-4, f.eks. \emph{Turqoise} og \emph{Sun Flower} eller \emph{Belize Hole} og \emph{Pomegranade}.
Disse gir myke overganger og kan oftest benyttes sammen og gi en myk og følsom overgang f.eks. mellom bakgrunnen i et grensesnitt og et element som kan eksempelvis være en knapp.

Til sist er det også mulig å benytte seg av to \textbf{komplementærfarger}.
Disse har vi plassert på skrå fra hverandre som \emph{Peter River} og \emph{Orange} eller \emph{Amethyst} og \emph{Sun Flower}. Disse fargen hvis kombinert tar hverandre helt ut og på slik måte dersom vi bruker dem hve siden av hverandre kan det brukes en tydelig markering. 
Generelt har vi selvsagt ikke benyttett oss av alle disse farger i hi-fi prototypen da mange av dem blir reservert til spesielle tilfellen. Kun mer monotone av farger i figur \ref{fig:farger} blir brukt til komponenter og bakrunner som ofte forekommer i komponentene.

\subsection{Open Source}
Vi har valgt å basere vår løsning kun på produkter som er tilgjengelige via open source eller som fri kode. Dette er ganske kritisk da prosjektet som kunne videreutvikles og ikke skal være avhengig av propriær teknologi.
I sin tur kommer prosjektet vårt til å være tilgjengelig åpnet slik at det kan opprettes så kallte <<forks>> av de som ønsker videreutvikle det i en annen rettning.