\section{Forbedringer}

\subsubsection{Bedre implementasjon av annonsesøk}
Vi fulgte oppgaveteksten ganske nøye fra begynnelsen av, og betalte kanskje litt for det mot slutten da vi så at søkemulighetene for boligsøker var litt begrenset.
Dette er likevel en veldig tidkrevende implementasjon som ville krevd mer tid enn vi har hatt tilgjengelig.

\subsubsection{Utvidet funksjonalitet i tabellen}
Tabellens implementasjon er veldig vellykket, men mulighetene en tabell tilbyr er også enorme. Vi valgt å bare ha 4 kolonner tilgjengelig for brukeren, men en kan se for seg at man kunne skjule "output-vinduet" og få en mye bredere tabell, som da også ville gitt bedre mulighet for å kunne editere objekter direkte i tabellen.
En kunne også se for seg filtrering direkte i tabellen i stedet for eller i tillegg til dagens søkefunksjon.

\subsubsection{Generell kommunikasjon i programmet}
Vi ga oss selv relativt store utfordringer da vi valgt MVC-arkitektur gjennom hele programmet samtidig som vi ønsket en tabell som responderer i det man gjør endringer. Vi har ikke hatt tid til å få tabellen til å respondere på absolutt alle endringer fra alle forskjellige vinkler. Med vinkler menes hvor man trykker for å få utført en operasjon. I det man klikker på "Ny Utleier"-knappen i toppanelet så utføres det fra en annen plass enn om en høyreklikker i tabellen og velger ny person.

Om mulig burde det lages en "<kommunikasjonssentral"> som alltid kjenner til og oppfatter alle hendelser. I det man er i et registreringsvindu så har man ikke kontakt med hovedprogrammet. Vi har løst dette så godt som mulig ved å sendes oppdateringshendelsen via et \texttt{interface}, som nok ville vært behov for uansett, men kanskje en helhetlig kommunikasjonsstruktur gjennom hele programmet ville vært lettere å jobbe med når en får ytterligere kompleksitet og utfordringer ettersom programmet utvikles.

\subsubsection{Bruk av SQL som datakilde}
Denne oppgaven ville vært ideell for lagring i SQL i stedet for å serialisere dataene. Det ville vært normalt med mye spørringer på tvers tabellene for å presentere statistikk, samt å lettere kunne skreddersy søkerutiner.

\subsubsection{Statistikk}
Tidsmangel gjorde at det ikke ble vektlagt noe utvidet statistikkpanel. Dette var noe vi ønsket å få til, men det vil være å anse som første prioritert ved eventuell videreutvikling av programmet.