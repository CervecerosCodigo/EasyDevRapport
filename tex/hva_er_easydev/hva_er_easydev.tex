\chapter{Hva er EasyDev?} \label{sec:hvaereasydev}

\lettrine[lines=2]{H}{} va er egentlig EasyDev? For å kunne fortelle hva EasyDev er må man gå tilbake et år i tid. Alle medlemmene i gruppen begynte sin utdanning på Informasjonsteknologi høsten 2013. Som relativt nye studenter skulle man begynne å utvikle websider og PHP-skript for websider. For å komme igang med slik utvikling må man kunne teste sin kode på en webserver hvilket er den eneste løsningen for å teste PHP-kode. 
Man har to muligheter: (1) kjøre koden på skolens webserver (alle studenter har sin egen bruker og filer som skal publiseres på nett legges i www-mappe) eller (2) sette opp en webserver på sin egen maskin.

Dersom man velger alternativ én krever det at man jobber direkte mot filene som finnes på serveren eller at man for hver gang som koden skal testes laster opp en ny versjon av filene til serveren. Det første alternativet er tregt og upraktisk og hvis man skal jobbe hjemmefra må det bruke VPN-tilkoblig, som i seg selv fungerer meget bra, men man savner mulighet for å samarbeide over slik løsning.

På grunn av disse praktiske begrensingene velger de fleste studentene å sette opp en server på sin egen maskin. Dersom man velger et slikt alternativ har man to valg: sette opp sin egen \textit{Apache} web-server med PHP-funksjonalitet eller gå for den enkleste og vanligste løsningen der man installerer en klone av Apache i form av XAMPP-applikasjon med et ferdig oppsett av PHP og webserver. Å sette opp sitt eget \textit{Apache} servermiljø er ikke noe som de fleste ønsker å sette seg inn i fordi det krever en del forkunnskaper og tid for å få serveren til å fungere som man ønsker. Dersom man skal bruke XAMPP gjøres det på en forhåndskonfigurert måte og servermiljøet kan ikke konfigureres fritt. XAMPP tillater ikke bruk av mange populære Apache plug-ins som mange av studentene ønsker ta i bruk etter som man utvikler ferdighetene i PHP og lignende teknologier.

Et annet aspekt er i forbindelse med gruppearbeid. I gruppearbeid skal flere personer samarbeide på ett sett av filer og dokument. Hvordan gjør man det? Den raskeste måten, som de fleste studenter velger, er at man setter opp en delt mappe mellom alle gruppemedlemmene på en av de mest populære synkroniseringstjenestene som \textit{Dropbox, Box, SkyDrive} eller \textit{Google Drive}. Det man raskt oppdager (oftest med negativt fortegn) er at man ganske så omgående vil bli offer for lagringskonflikter. Hvis to personer har åpnet en og samme fil samtidig kommer synkroniseringstjenesten til å opprette to versjoner av filen på grunn av lagringskonflikt. 
Men det er ikke alltid nødvendig at filen skal være åpen av to personer samtidig for å forårsake én konflikt. For eksempel, om ikke en endring blir synkronisert etter at en person er ferdig med filen og neste person begynner å endre samme fil før synkroniseringen er fullført så blir den første personens endringer borte. 

Synkroniseringstjenester er tross alt et meget bra verktøy så lenge flere brukere ikke skal editere de samme filene samtidig. Det eneste alternativet som finnes er å starte bruk av et versjonskontrollsystem som f.eks. git, cvs eller subversion. Versjonskontrollsystem er et fantastiske verktøy men det alle disse systemene har til felles er at de har en meget bratt læringskurve. Dette er noe som ofte skremmer nye brukere og det er vanskelig å motivere alle gruppens medlemmer til å faktisk bruke tid på å lære seg et nytt komplekst verktøy som i tillegg må brukes via terminal\footnote{Det finnes mange brukergrensesnitt spesielt for git, men disse viser seg ofte å være vanskeligere å bruke enn den terminalbaserte løsningen. Bruk av versjonshåndtering via terminal gir garantert best kontroll og brukeropplevelse.}.

Andre utfordringer oppstår da man for første gang har behov å sette opp en egen database. I skolen er det vanlig at man jobber med \textit{MySQL} databasesystem. Det man får tilbudt er ferdig konfigurert, der man ikke har muligheter til å slå på tilleggsfunksjoner man gjerne trenger å bruke i sitt arbeid. Det er også vanlig å bruke \href{http://en.wikipedia.org/wiki/PhpMyAdmin}{\textit{phpMyAdmin}} som er et webbasert grensesnitt til konfigurasjon av en \textit{MySQL} server. Grensesnittet krever dog at man først skal ha konfigurert både webserver og php for at dette skal fungere.

Dette er kun to eksempel på frustrasjoner som møter en ny student på en IT-utdanning. Det er ikke bare pensum og lesestoff men man må også lære seg mestre flere verktøy som er nødvendige for å komme igang med både gruppearbeid og sin egen utvikling. Så hvis en ser tilbake på det spørsmålet som egentlig ble stilt i begynnelsen av dette kapittelet: Hva er EasyDev? Da kan det egentlig besvares ganske enkelt: 

\begin{quotation}
\emph{\textbf{EasyDev} er et brukergrensesnitt til en virtuell Linux maskin som inkluderer alle verktøy og moduler som en student på en IT-utdanning har behov for. Derfor, EasyDev er et system som skal gjøre det mulig å komme igang med skolearbeid og utvikling uten å lese en linje av kjedelig dokumentasjon og istedet fokusere på å skape.}
\end{quotation}


\section{Målgruppe} \label{sec:målgruppe}
Innledningsvis begrenses målgruppen til nye IT studenter ved universitet og høgskoler. Ettersom systemet skal følge Linux programmeringsstil\cite{book:unixprog} vil det være mulig å skalere det etter egne behov. Det kan på et senere tidspunkt bli aktuelt å sette opp et system som kun er rettet mot andre utdanninger, f.eks. kun med støtte for meldingssystem og gruppearbeid.

\section{Eksisterende løsninger}
Undersøkelsene som ble gjort i forbindelse med prosjektet fant ikke noe system som gir mulighet for akkurat den funksjonaliteten som denne løsningen tilbyr. Det finnes system som \href{http://en.wikipedia.org/wiki/Webmin}{\textit{webmin}} men det er mer rettet mot systemadministratorer og tilbyr ingen <<desktop>>-funksjonalitet. Dette er en skreddersydd løsning som til å begynne med er rettet mot kun IT-studenter, og gir samarbeidsløsninger som vi ikke har funnet andre steder i markedet i dag.

\section{Nyskapende}
Det er flere aspekter man vil legge merke til som dette prosjektet tilbyr og som er helt nyskapende og som det ikke er funnet i noe liknende system av denne typen. Listen over funksjonalitet er ganske lang, og derfor vil rapporten kun beskrive de viktigste delene. For en utfyllende beskrivelse av alle de modulene som er kartlagt og planlagt, henvises leser til appendiks \ref{app:funksjonalitet}, side \pageref{app:funksjonalitet}.


\begin{description}
\item[Modulbasert] Ingen av modulene som brukes til konfigurering av systemet er hardkodet i EasyDev. Alt er modulbasert, det medfører at man kan enkelt legge til eller fjerne pakker etter eget ønske. Hver skole eller avdeling som tar i bruk EasyDev kan selv velge hvilke pakker og moduler som skal tas i bruk. Det gir også mulighet for utvikling av egne moduler. Brukere som ønsker spesifikk funksjonalitet kan utvikle egne moduler mot EasyDev sitt API. 
\item[Laget for bruk i skolen] Det at man kan få et ferdigdefinert system, tilpasset av hver skole, gjør at det er veldig fleksibelt fra skolen sitt synspunkt, samtidig gir det hver enkelt student mer råderett over det systemet han/hun har fått tildelt. Det at man samtidig kan invitere medstudenter til å jobbe på sitt eget system gjør at gruppearbeid forenkles. Det er ment at EasyDev skal gi den beste mulige løsningen for alle parter som er involvert.
\item[Enkel videreutvikling]
Ettersom løsningen bygger på at man har et virtuelt operativsystem som backend kan den egentlig videreutvikles etter eget ønske. Et eksempel på dette kan være at man tilpasseer den til å kjøre på en meget rimelig hardware som \textit{Rasberry-Pi} og brukes på billige skolemaskiner i utviklingsland.
\end{description}