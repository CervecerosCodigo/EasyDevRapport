\chapter{Hva er EasyDev?} \label{sec:hvaereasydev}

\lettrine[lines=2]{H}{}va er egentlig EasyDev? For å kunne fortelle hva EasyDev er må vi egentlig først gå tilbake et år i tid. Alle medlemmer i gruppen vår begynte sin utdanning på Informasjonsteknologi høsten 2013. Som relativt nye studenter skulle vi begynne å utvikle websider og PHP skript for websider. For å komme igang med slik utvikling må man kunne teste sin kode på en webserver ettersom dette er den eneste løsningen for å teste PHP-kode. Man har to muligheter, (1) kjøre koden på skolens sin webserver (alle studenter har sin egen bruker og filer som skal publiseres på nett legges i www-mappe) eller, (2) sette opp en webserver på sin egen maskin. \\
Dersom man velger alternativ én krever det at man jobber direkte mot filene som finnes på serveren eller at man for hver gang som koden skal testes laster opp en ny versjon av filene. Det første alternativet er tregt og upraktisk og hvis man skal jobbe hjemmefra må det bruke VPN-tilkoblig som i seg selv fungerer meget bra, men man har ikke mulighet for å samarbeide på denne løsningen.\\
På grunn av slike praktiske begrensinger velger de fleste studentene å sette opp en server på sin egen maskin. Dersom man velger  det alternativet har man to valg: sette opp sin egen Apache web-server med PHP-funksjonalitet eller gå for den enkleste og vanligste løsningen der man installerer en klone av Apache i form av XAMPP-applikasjon med en ferdig oppsett av PHP og webserver. Å sette opp sitt eget Apache servermiljø er ikke noe som de fleste ønsker å sette seg inn i ettersom det krever en del forkunnskaper og tid for å få serveren til å fungere som man ønsker. Dersom man skal bruke XAMPP må dette gjøres på en forhåndskonfigurert måte og servermiljøet kan ikke konfigureres fritt. XAMPP tillater ikke bruk av mange populære Apache plug-ins som noen av studentene ønsker ta i bruk ettersom man utvikler ferdighetene i PHP og lignende.

Neste begrensning kommer i forbindelse med gruppearbeid. I gruppearbeid skal flere personer skal samarbeide på et sett av filer og dokument. Hvordan gjør man det? Den raskeste måten som de fleste studenter velger er at man setter opp en delt mappe mellom alle gruppemedlemmer på en av de mest populære synkroniseringstjenestene som Dropbox, Box, SkyDrive eller Google Drive. Det som man raskt oppdager, dessverre på den sure måte, er at man vil ganske så omgående bli offer for lagringskonflikter. Hvis to personer har åpnet en og samme fil samtidig kommer synkriniseringstjenesten til å opprette to versjoner av filen på grunn av lagringskonflikt. Men det er ikke alltid krav om at filen skal være åpen samtidig, feks dersom synkroniseringen ikke alltid blir kjørt etter at en bruker har lagret å lukket filen sin. En annen bruker kan allerede åpnet filen før synkroniseringen ble ferdigkjørt fra bruker som sist gjorde endringer i filen. I enkelte tilfeller kan en synkronisering være flere minutter forsinket. I vårt tilfelle har vi selv fått håndtere store filkonflikter og tap av flere timers gruppearbeid. Synkroniseringstjenester er et fantastisk verktøy så lenge flere brukere ikke skal editere filer samtidig. Det eneste alternativet som finnes er å starte bruk av et versjonskontrollsystem som f.eks. git, cvs eller subversion. Versjonskontrollsystem er et fantastisk verktøy men det alle disse systemene har til felles er at de har en ganske så bratt læringskurve. Dette er noe som ofte skremmer nye brukere og det er vanskelig å motivere alle gruppens medlemmer til å faktisk bruke tid på å lære seg et nytt kompleks verktøy som ofte må brukes via en terminal (det finnes mange brukergrensesnitt spesielt for git, men disse viser seg ofte å være vanskeligere å bruke enn den terminalbaserte løsningen).

Andre utfordringer oppstår da man for første gang har behov å sette opp en egen database. I skolen er det vanlig at man jobber med MySQL databasesystem. Det man får tilbudt er ferdig konfigurert, der man ikke har muligheter til å slå på tilleggsfunksjoner man gjerne trenger å bruke i sitt arbeid. Det er også vanlig å bruke phpMyAdmin som er et webbasert grensesnitt til konfigurasjon av en MySQL server. Grensesnittet krever dog at man først skal ha konfigurert både webserver og php for at dette skal fungere.

Disse er kun to eksempel på frustrasjoner som møter en ny student på en IT-utdannese, det er ikke bare pensum og lesestoff men man må også lære seg mestre flere verktøy som er nødvendige for å komme igang med både gruppearbeid og sin egen utvikling. Så hvis vi ser tilbake på det spørsmål som egentlig ble stil i begynnelsen av dette kapittelet: Hva er EasyDev? Så kan det egentlig besvares ganske enkelt: EasDev er et brukergrensesnitt til en virtuell Linux maskin som inkluderer alle verktøy og moduler som en student på en IT-utdanning har behov for. Med enkle ord, EasyDev er et system som skal gjøre det mulig å komme igang med skolearbeid og utvikling uten å lese en linje av kjedelig dokumentasjon og istedet fokusere på å skape.

\section{Målgruppe} \label{sec:målgruppe}
Innledningsvis ser vi for oss at målgruppen begrenses til nye IT studenter ved universitet og høgskoler. Ettersom systemet skal følge Linux programmeringsstil\cite{book:unixprog} vil det til å være mulig å skalere dette etter egne behov. Det kan på seinere et tidspunkt bli aktuelt å sette opp et system som kun er rettet mot andre utdanninger, f.eks. kun med støtte for meldingssystem og gruppearbeid.

\section{Eksisterende løsninger}
Så langt vi kjenner til finnes det ikke noe system som gir mulighet for akkurat den funksjonaliteten som vår løsning tilbyr. Det finnes system som \href{http://en.wikipedia.org/wiki/Webmin}{webmin} men det er mer rettet mot systemadministratorer og tilbyr ingen <<desktop>>-funksjonalitet. Dette er en skreddersydd løsning som til å begynne med er rettet mot kun IT-studenter, og gir samarbeidsløsninger som vi ikke har funnet andre steder i markedet i dag.

\section{Nyskapende}
Det er flere ting som vi kan observere at systemet vårt gjør som er helt nyskapende og som vi ikke kan finne i noe liknende system av den typen. Listen over funksjonalitet er ganske lang, og derfor velger vi å kun beskrive de viktigste delene i dette avsnitt. For en utfyllende beskrivelse av alle de moduler vi så langt har kommet på henviser vi leser til appendiks \ref{app:funksjonalitet} side \pageref{app:funksjonalitet}.


\begin{description}
\item[Modulbasert] Ingen av modulene som brukes til konfigurering av systemet er hardkodet i EasyDev. Alt er modulbasert, det medfører at man kan enkelt legge til eller fjerne pakker etter eget ønske. Hver skole eller avdeling som tar i bruk EasyDev kan selv velge hvilke pakker og moduler som skal tas i bruk. Det gir også mulighet for utvikling av egne moduler. Brukere som ønsker spesifikk funksjonalitet kan utvikle egne moduler mot EasyDev sitt API. 
\item[Laget for bruk i skolen] Det at man kan få et ferdigdefinert system, tilpasset av hver skole, gjør at det er veldig fleksibelt fra skolen sitt synspunkt, samtidig gir det hver enkelt student mer råderett over det systemet han/hun har fått tildelt. Det at man samtidig kan invitere medstudenter til å jobbe på sitt eget system gjør at gruppearbeid forenkles. Det er ment at EasyDev skal gi den beste mulige løsningen for alle parter som er involvert.
\item[Enkel viderutvikling]
Ettersom løsningen bygger på at vi har et vituelt operativsystem som backend kan den egentligen videreutvikles som vi ønsker. Et eksempel på dette kan være at man gjør den tilpasset å kjøre på en meget rimmelig hardvare som \textit{Rasberry-Pi} og brukes på billege skolemaskiner i utviklingslender.
\end{description}