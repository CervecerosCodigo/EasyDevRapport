\chapter{Hva er EasyDev?}

\lettrine[lines=2]{H}{}va er egentligen EasyDev? For å kunne fortelle hva EasyDev er må vi egentlig først gå tilbake tid et år. Alle medlmmer i gruppen vår begynte sin utdanning på informasjonsteknologi for cirka et år siden. Som relativ nye studenter skulle vi beguynne å utvikle websider og PHP skript for websider. For å komme igang med slik utvikling må man kunne teste sin kode på en webserver ettersom dette er den eneste løsningen for å teste PHP kode. Det gjør at man har i grunn og bunn to muligheter (1) kjøre koden på skolens sin webserver (alle studenter har sin egen bruker og filer som skal publiseres på nett legges i www mappe) eller (2) sette opp en webserver på sin egen maskin. Dersom man velger alternativ én krever det at man jobber direkte mot filene som finnes på serveren eller at man for hver gang som koden skal testes laster opp en ny versjon av filene. Hvilke subløsning av disse to man velger vil skape ventetid og ganske treg respons ved lagring av filer, spesielt hvis man lagrer filene direkte mot serveren. Hvis man skal jobbe hjemmefra må det bruke VPN tilkoblig som for seg fungerer meget bra men vi kommer ikke unna med fysiske begrensninger som avstand hvilket medfører ytterliggere ventetider. 
På grunn av slike praktiske begrensinger velger mesteparten av studenter å sette opp en server på sin egen maskin. Dersom man velger slik alternativ har man i grunn og bunn to valg: sette opp sin egen Apache web server med PHP funksjonalitet eller gå for den seineste løsningen der man installerer en klone av Apache i form av XAMPP applikasjon med en ferdi oppsett av PHP og webserver. Å sette opp sin egen apache servermiljø er ikke noe som de fleste ønsker å settte seg inn i ettersom det krever en del forkunskaper og tid for å få servern til å fungere som man ønsker. Dersom man skal bruke XAMPP må dette gjøres på en spesifikk måte og servermiljøen kan ikke konfigureres fritt fesk. dersom man ønsker å tilpasse denne for gruppearbeid eller liknende. XAMPP tillater ikke bruke av mange populære Apache plug-ins som mer noen av studentene ønsker å bruke i takt at man har blitt flinkere med webutvikling.

Neste begrensning kommer i samband da man skal starte med gruppearbeid. Gruppearbeid medfører at man flere personer skal samarbeide over et sett av filer og dokument. Hvordan gjør man det? Den raskest måten som de fleste studenter velger er at man setter opp en delt mappe mellom alle gruppemedlemmer på en av de mest populære synkroniseringstjenestene som DropBox, Box, SkyDrive eller Google Drive. Det som man vil rask oppdage på dessverre den bittre måten er at man vil bli ganske så omgående bli offer for lagringskonflikter. Hvis to person har åpnet en og samme fil samtidig kommer synkriniseringstjenesten til å opprette to versjoner av filen på grunn av lagringskonfilkt. Men det er ikke altid krav på at filen skal være åpne samtidig ettersom det ikke aldtid synkroniseringen ble kjørt etter at en bruker har lagret å lukket filen sin. En annen bruker kan alerede åpnet filen før synkroniseringen ble ferdikjørt fra bruker som sist gjorde endringer i filen. I enkelte tilfelde kan en synkronisering være flere minutter forsinket. I vårt tilfelde har vi selve fått håndtere stoer filkonfikter og tap av flere timers gruppearbeid. Synkroniseringstjenester er et fantastisk verktæy så lenge flere brukere ikke skal editere filer samtidig. Det eneste alternativet som finnes er å starte bruk av et versjonskontrollsystem som f.eks. git, cvs eller subversion. Versjonskontrollsystem er et fantastisk verktøy men det som alle disse system har til felles er at de ha en ganske så bratt læringskurve. Dette er noe som ofte skremmer nye brukere og er vanskelig å motivere alle gruppens medlemmer til å fanktisk bruke tid på å lære seg et nytt kompleks verktøy som ofte må brukes via en terminal (det finnes mange brukegrensesnitt spesielt for git men disse viseg seg oftest vanskeligere å bruke enn den  terminlbaserte løsningen).

Neste vanskeligheter kommer da man for første gang er i behov å sette opp en egen database. I skolen er det vanlig at man jobber med MySQL databasesystem. Hvirdan man enn velger å setter opp databasen stiller det krav på at man må konfigurere databasen manuelt hvilket krev at brukeren setter seg inn i systemets dokumentasjon. Det er vanlig at man bruker et grensesnitt phpMyAdmin som er et webbasert grensesnitt til konfigurasjon av en MySQL server. Grensesnittet krever dog at man først skal ha konfigurert både webserver og php for at dette skal fungere.

Disse er kun to eksempel på frustrasjoner som møter en ny student på en IT udannese, det er ikke bare pannsun og lesestoff men man må også lære seg bemestre flere av verktøy som er fulstendig nødvendige for å komme igang med både gruppearbeid og sin egen utvikling. Så hvis vi ser tilbake på det spørsmål som egentlig ble stil i begynnelsen av dette kapittel: Hva er EasyDev? Dette kan egentlig besvares ganske enkelt: EasDev er et brukerinterface til en virtuell Linux maskin som innkluder alle verktøy og moduler som en stundent på en IT utdanning har behov. En mer detaljert beskrivelse av systemet følger i underliggende avsnitt av dette kapittel. Med enkle ord EasyDev er et system som skal gjøre det mulig å komme igang med skolearbeid og utvikling uten å lese en linje av kjedelig dokumentasjon og istende fokusere på skapenede.

\section{Målgruppe} \label{sec:målgruppe}
Innledningsvis ser vi for oss at målgruppen begrenses til nye IT studenter ved universitet og høgskoler. Ettersom systemet skal følge Linux programmeringsstil\cite{book:unixprog} kommer det til å være mulig å skalere dette etter egne behov. Det kan på seinere tidspunnkt bli aktuelt å sette opp et system som kun er riktet mot andre utdanninger og som kun omfatter f.eks. kun støtte for meldingssystem og gruppearbeid.

\section{Eksisterende løsninger}
Så langt vi kjenner til finnes ikke noen system som gir mulighet til akkuratt samme type av funksjonalitet som vår løsning. Det finnes system som \href{http://en.wikipedia.org/wiki/Webmin}{webmin} men er mer rettet for systemadministratorer og tilbyr ingen <<desktop>> funksjonalitet. Dette er en skreddersydd løsning som er til å begynne med rettet mot kun IT studenter og derfor ser vi ikke at noen liknende idag eksisterer på markedet. 

\section{Nyskapende}
Hva gjør vi som gjør systemet vårt nyskapende?