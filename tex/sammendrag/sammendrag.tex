\begin{abstract}
\lettrine[lines=3]{E}{asyDev} er verktøy enhver IT-student burde ha tilgang til. Enkelt tilgjengelig via en nettleser gir mulighet til enkel administrasjon av webområder, SQL-databaser og deleing av filer mellom gruppemedlemmer. Løsningen er basert på \textit{Linux}, \textit{Apache} web server og \textit{MySQL} database der hver student får tildelt en virtuell maskin. 
I den virtuelle maskinen kan en opprette, endre og slette databaser samt installere nødvendige moduler for arbeidet sitt. 
Produktet har et utvalg av nettbaserte editorer for \textit{SQL}-spørringer, \textit{HTML} og \textit{Javascript}. 
Alt dette er satt sammen i et system med tiltalende grensesnitt som gjør at studenter raskt kan komme i gang med utvikling og skolearbeid.
Rapporten beskriver prosess og besluttinger som var tatt underveis  og ligger til grunn for utforming av en prototype for EasyDev. 
I tillegg var det også gjennomført brukertester for prototypen.
Resultat fra brukertestene gav en indikasjon på at både idé og utforming gir en god balanse mellom brukeropplevelse og funksjon som resulterer i et balansert produkt etterspurt av studentene.



\end{abstract}