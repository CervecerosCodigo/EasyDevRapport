\chapter{Evaluering}
\lettrine[lines=2]{P}{} roduktet vi sitter igjen med etter alt arbeidet er utført er noe annerledes enn det vi så for oss da vi begynte å jobbe med prosjektet. Ambisjonene våre var rettet mot å lage et så godt produkt som mulig, samt å implementere noe funksjonalitet i dybden for å gi brukeren en følelse på hvordan sluttproduktet kunne se ut og hvordan det ville være å jobbe med det. Dette kapittelet vil berøre alle aspektene ved produktet, hvorfor det ble som det ble og hva vil ville ha gjort annerledes.

\section{Produktet}
Resultatet av HiFi-prototypen er det som vi anser som produktet vårt. Den er dog en dårlig pekepinn på hva prosjektet går ut på og dermed vil denne rapporten være det viktigste bidraget til sluttproduktet vårt. Uten rapporten og historien som følger med rundt beslutningene som ble gjort og prosessen underveis vil ikke prototypen ha noen verdi. Prototypen må derfor sees i lys av det man leser i rapporten.


\section{Begrensninger} \label{sec:begrensninger}
Dette punktet har vært den største utfordringen vi har støtt på. I et prosjekt der man skal lage noe ønsker man å har et produkt man kan være fornøyd med, og som vi kunne tenke å jobbe med selv. Dette var jo målet vi hadde satt oss også, men det ble klart ganske tidlig at vi måtte omprioritere, lage ny fremdriftsplan og sette nye mål på hva vi kunne få til innenfor angitt tid.
Den største utfordringen det medførte var i hvor stor grad vi skulle implementere funksjonalitet. Om vi hadde hatt tid kunne vi implementert veldig mye av funksjonaliteten, i alle fall i brukergrensesnittet, men det har altså ikke vært mulig innenfor tidsrammene.
Resultatet har da blitt at vi videreutvikler mockup/prototype fra første innlevering og viser prototypen i "ny drakt" slik at man får en bedre, mer funksjonell prototype, men som ikke har mer funksjonalitet implementert.\\
Vi har prøvd å forholde oss til prinsipper for MMI-faget. Først og fremst har vi jobbet med farger som passer godt sammen, samt laget en oversiktlig meny som gir brukeren rask tilgang til de forskjellige kategoriene av funksjonalitet vi har ønsket å implementere.
Da produktet vårt er av en teknisk art og ikke ment for den jevne bruker vil ikke alle menypunkter være selvforklarende, men at man har en liten "læringskurve" for å bli vant med produktet slik at man kan utnytte all den funksjonalitet produktet tilbyr.
Det skal likevel ikke være behov for å kunne alle disse avanserte funksjonene for at man kunne bruke systemet på en tilfredsstillende måte.


\section{Hva ville vi ha gjort annerledes?}
Premisset for prosjektoppgaven i MMI er annerledes enn det vi la opp til da vi kom opp med vår idé, som var før oppgaven hadde blitt delt ut. Dette har medført at vi ikke fikk tid til å utvikle løsningen vår slik vi ønsket i utgangspunktet, men i vi stedet tilpasset produktet vårt til oppgavens spørsmål og krav.
Det medførte til at vi laget et forslag mer som en konseptløsning med god dokumentasjon som viser ideene og ambisjonene våre i stedet for å lage et best mulig sluttprodukt med mest mulig funksjonalitet. Vi ville likevel stå i fare for å ikke ha tid til å gjøre ferdig produktet eller komme med en god nok løsning som rettferdiggjorde valgene vi hadde tatt.\\
Videre kunne vi tenkt oss å tatt med en brukerundersøkelse før vi begynte arbeidet for å se hva andre studenter kunne se for seg for løsninger og funksjonalitet. Det ville ikke påvirket arbeidet vårt i nevneverdig grad, men det ville vært interessant å ta med flere av de resultatene inn som tiltenkt funksjonalitet i produktet.
