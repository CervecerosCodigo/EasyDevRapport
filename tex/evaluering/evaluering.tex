\chapter{Evaluering}
\lettrine[lines=2]{P}{}roduktet som er igjen etter alt arbeid er utført er noe annerledes enn det som var planlagt ved begynnelsen av prosjektet.
Ambisjonene var rettet mot å lage et så godt produkt som mulig, samt å implementere noe funksjonalitet i dybden for å gi brukeren en følelse på hvordan et sluttprodukt kunne se ut og kan brukes. 
Dette kapittelet berører alle aspektene for produktet, hvorfor det ble som det ble og hva vil ville ha gjort annerledes.

\section{Produktet}
Resultatet av hifi-prototypen er det som vi anser som produktet vårt. 
Den er dog en dårlig pekepinn på hva prosjektet går ut på og dermed vil denne rapporten være det viktigste bidraget til sluttproduktet vårt. 
Uten rapporten og historiken som følger med rundt beslutningene som ble gjort under prosessen underveis vil ikke prototypen ha noen verdi. Prototypen må derfor sees i lys av det man leser i rapporten.

\section{Begrensninger} \label{sec:begrensninger}
Dette punktet har vært den største utfordringen vi har støtt på. I et prosjekt der man skal lage noe ønsker man å har et produkt man kan være fornøyd med, og som vi kunne tenke å jobbe med selv. Dette var jo målet vi hadde satt oss også, men det ble klart ganske tidlig at vi måtte omprioritere, lage ny fremdriftsplan og sette nye mål på hva vi kunne få til innenfor angitt tid.

Den største utfordringen var i hvor stor grad hvordan selve funksjonaliteten skal implementeres. 
Gitt mer tid kunne det implementeres mer av funksjonaliteten, fremfor alt i brukergrensesnittet, noe som ikke vært mulig innenfor gitt tidsrammene.
Resultatet baserer seg på en videreutvikling mockup/prototype fra første innlevering og viser prototypen i "ny drakt". Dette gir en bedre og mer funksjonell prototype men som ikke har mer funksjonalitet implementert.

Under den kreative prosessen var det pålagt en strikt avgrensing for at resultatet skal være innenfor prinsippene for MMI-faget.
Først og fremst jobbedes det med farger som passer godt sammen og utforming av en oversiktlig meny. En meny som gir brukeren rask tilgang til de forskjellige kategoriene av funksjonalitet vi har ønsket å implementere.
Da produktet er av en teknisk art og ikke ment for den jevne bruker vil ikke alle menypunkter være selvforklarende, men at det finnes en liten <<læringskurve>> for å bli vant med produktet og frigjøre utnytte all dess funksjonalitet.
Fra andre siden skal det ikke heller være behov å kunne alle de avanserte funksjonene for at bruke systemet på en tilfredsstillende måte.

\section{Hva ville vi ha gjort annerledes?}
Premisset for prosjektoppgaven i MMI er annerledes enn den som innledningsvis ble tatt frem av prosjektgruppen.
Noe som kanskje har medført at tiden ikke strakk til å utvikle samme konsept som var presentert i første utkast men ble tilpasset til oppgavens spørsmål og krav.

Det medførte til at det ble laget et forslag som er mer av en konseptløsning med god dokumentasjon. Der ideene og ambisjonene viser mulighetene til best mulig sluttprodukt med mest mulig funksjonalitet. 
Det var tross alt noe risiko der konsekvensen var at prototypen skulle forbli et uferdig produkt.

Selve utformingsprosessen kunde også startes med en spørreundersøkelse for å kartlegge hvilken funksjonalitet eller løsninger som ønskes av studentene. 
Dette hadde eventuelt ikke påvirket selve kreative prosessen men det ville vært interessant å ta med flere av de resultatene inn som tiltenkt funksjonalitet i produktet.

Det var også diskutert om hvordan gjøre eventuelle forbedringer i grensesnittet. Det finnes flere alternativer der det ble drøftet mulighet over å bruke litt forskjellige farger på de ulike komponetene. 
Tanken var å gruppere kompoenentene gjennom å bruke farger i tilleg til formvariasjoner. 
På tråd med gestalt prinsippene vile da kompoenentene bli gruppert eller skilt som forskjellige elementer med hjelp av analoge- eller komplementærfarger eller gruppert med nyasnvariasjoner i tonenyanser. 
Det hadde vært ønskeverdt å implementere slik funksjonalitet frem til brukertestingen for å teste ut hvis det gir noen forskjeller.
Slik brukerteting borde da blir gjennomført på to grupper, der ene gruppen fikk for seg presentert den tidligere løsningen og andre gruppe som fikk teste produktet med den nye utformingen. Deretter kunde de settes opp en hypotese over hvilken av de to løsniger som er mest fordelaktig for sluttbrukeren.

\section{Veien videre}
EasyDev føddes som en spontan idé for gruppeprosjekt i MMI faget. 
Dersom man ser det frå et studentperspekitv er det en god idé som prosjektmedlemmer eventuelt ønsker å bygge videre på. 
Dette er et produkt som ikke finnes i markedet i den form som er spesifisert i oppgaven. 
Det er fult mulig å forfine produktet og i tillegg lansere det utenfor skoleverden. 
Det kan drøftes muligheter for en komersiell lasnering i f.eks. \textit{skyen}. 
Det vil dog forutsette flere tilleg og eventuelt en modul med fullverdig dektop løsning i \textit{skyen}. 
Slik desktop er ikke noen nytt og er alerede tilgjengelig. 
Idag ser vi oftest tre trender innen <<\textit{clound computing}>>: \textit{SaaS - software as service}, \textit{PaaS - platform as service} eller \textit{DaaS - desktop as service}. 
Det er ingen eller få muligheter å få både et operativsystem, desktop og server i form av en service. 

I vår løsning tilbyr man i tillegg mange verktøy og funksjonalitet som nærmere beskrives i appendiks \ref{app:funksjonalitet} (side \pageref{app:funksjonalitet}). 
Noe som vil gi en meget unik kombinasjon dersom man legger til dektop funksjon i produktet.
Veien videre borde begynne med at man gjennomfører en grunnlig markedsundersøkelse med hensikt å finne hvis slik etterspørsel finnes i det globale markedet. 
Deretter må gjennomføres kartlegging av produktets styrker, svakheter, muligheter og trusler med hensikt å finne ut produktets konkurransestyrke og potensial.