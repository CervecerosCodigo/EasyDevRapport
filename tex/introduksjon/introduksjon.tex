\chapter{Introduksjon}

\section{Om rapporten}
Rapporten består av flere kapitler som kan leses hver for seg og som har hvert sine formål. Det er lagt stor vekt på dokumentasjon av beslutninger, og prosessen underveis. 
De viktigste punktene i rapporten er de som omhandler tolkning av oppgaven (avsnitt \ref{sec:tolkning}), avgrensningene som ble gjort (avsnitt \ref{sec:avgrensninger} og \ref{sec:begrensninger}), hvorfor det be gjort som det ble gjort osv. 

\begin{description}

\item[Introduksjonen] Går gjennom litt av forutsetningene for oppgaven, målene som ble satt, tolkningen av oppgaven og valgene som ble tatt på bakgrunn av det. 

\item[Hva er EasyDev?] En introduksjon til produktet. Gir kort beskrivelse av målgruppe og hvordan det er tenkt at produktet skal brukes.

\item[Prosessdokumentasjon] Her beskrives det hvordan det har blitt jobbet med utvikling av prototypene og hvilke valg som har blitt tatt underveis.

\item[Brukertesting] Brukertest av hi-fi prototype på noen relevante brukere og en kort oppsummering av resultatene.

\item[Evaluering] Ser her med et kritisk øye på arbeidet. Diskuterer selve produktet, begrensningene og hva som kunne ha vært gjort annerledes hvis...

\end{description}



\section{Tolkning av oppgaven} \label{sec:tolkning}
Oppgaven skulle egentlig besvare følgende spørsmål: 
\begin{quote}
\textit{Et	blikk	inn	i	framtida:	Hvordan	kan	vi	om	10-20	år	bruke	teknologi	
– både	dagens	og	tenkt	framtidig	– til	å	gjøre	livet	bedre	for	
mennesker	som	har	spesielle behov.}
\end{quote}

Da prosjektoppgaven ble delt ut hadde gruppen allerede tenkt mye på et interessant prosjekt den kunne jobbe med og som passet inn i MMI-faget. Det viste seg at oppgavene som ble delt ut var noe annerledes enn det som gruppen hadde lagt forutsatt.\\

Som studenter i det andre året på høgskolen vet gruppemedlemmene veldig godt hvordan det er å begynne på utdanningen og det å bli kastet inn i en verden av nye verktøy, konfigurasjon av egne maskiner og lignende for å få tilgang til skolens ressurser. Prosjektet gikk derfor lage noe som ikke finnes fra før men kan hjelpe studenter med å komme i gang med pensum i stedet for å bruke tid på å sette opp et fungerende utviklingsmiljø.

Prosjektet ble da til en portal for alle IT-studenter ved høgskolen som tilrettelegger for økt læring og gir bedre verktøy for samarbeid og gruppearbeid er et fullgodt alternativ til den oppgaveteksten som ble tildelt. Det ble selvsagt avklart med lærer før det ble satt i gang med arbeidet. 

\section{Formål}
Å lage et system for nye studenter som forenkler arbeidet og mulighetene for å sette igang med utvikling og å jobbe sammen i gruppearbeid uten å bruke tid på å studere manualer og konfigurasjon av systemet. Dagens løsning fungerer dårlig da en ikke kan konfigurere miljøet man jobber mot, og ikke har mulighet for å jobbe sammen i gruppe på én brukers system. Les mer om det i avsnitt \ref{sec:hvaereasydev}.

\section{Mål}
%Gi et nytt innspill på hvordan en institusjon som HIOA kan organisere og implementere verktøyene de tilbyr IT-studenter i årene fremover. 
%Målet er videre at det vi har kommet frem til i prosjektet skal kunne realiseres om en har tid og ressurser. Det er da følgelig ingen sci-fi løsning der man er avhengig av fremtidig utvikling på andre plan for å kunne implementere vårt produkt.
\begin{itemize}
\setlength{\itemsep}{1pt}
\setlength{\parskip}{0pt}
\setlength{\parsep}{0pt}
\item Komme opp med produktidéer. Velge en av disse idéer til å lage utkast.
\item Lage mockups (low-fi prototype) som viser systemet basert på den valgte idéen.
\item Videreutvikle alle tilgjengelige mockups av skjermbilder til en hi-fi prototype som representerer første utkast over systemets grensesnitt.
\end{itemize}

\section{Avgrensninger} \label{sec:avgrensninger}

\subsection{Hva skal prosjektet ikke levere}
Prosjektet skal ikke resultere i et ferdig produkt. Det er ikke et mål med kurset eller prosjektoppgaven. Grunnet tiden var det ikke mulighet til å implementere arkitektur og utvikle et fungerende produkt.

\subsection{Hva skal prosjektet levere}
Prosjektet skulle resultere i en lo-fi og en hi-fi prototype for å visualisere hvordan det var tenkt at produktet skal fungere. Prototypen skulle gi en <<\textit{technology preview}>> over noen funksjonene som var tenkt at skulle implementeres.
Hovedfokuset ble likevel lagt på rapporten ettersom det er der man finner informasjon om prosjektets visjoner, hvilke valg som ble tatt, prosessen underveis og gruppens egen evaluering av produktet. I tillegg ble det laget en poster som presenterer løsningen.

\subsection{Hvordan er prosjektet avgrenset}
Følgende avgrensninger ble satt opp for å gjøre det enklere å avgrense oppgaven:
\begin{description}
\item[Tid] Tilgjengelig tid gir automatisk avgrensning i forhold til hvor konkret implementasjonen kan bli.
\item[Teknologi] Definere hvilken teknologi man ønsker at produktet skal basere seg på.
\item[Målgruppe] Hvilken målgruppe jobber man mot?
\item[Fremtid] Er løsningen fremtidsorientert? Finnes en andre produkter som gir samme funksjonalitet?
\item[Faglig utfordring] Er prosjektet stort nok, eller for stort?

\end{description}

\section{Hjelpemidler og verktøy} \label{sec:hjelpogverkt}
Underveis i prosjektet ble det brukt \textit{Google Drive} og \textit{GitHub} for utveksling av kildekode. Mockups til første prototype ble opprettet via \href{https://balsamiq.com/}{\textit{balsamiq mockups}} plugin til \textit{Google Drive}. \textit{Pecha Kucha} presentasjon er opprettet med \href{https://prezi.com/z_1ipfrf4m_3/skolelinux-pa-ny-mate/}{\textit{Prezi}}. Viktigste UI elemeter til hi-prototypen er laget med \href{http://getbootstrap.com/}{bootstrap} og utformet med hjelp av med \href{http://designmodo.github.io/Flat-UI/}{FlatUI}. Til utvikling av kildekode er det brukt \textit{Eclipse IDE} med \textit{JavaScript}, \textit{Web}, og \textit{PHP}-eclipse miljø. Hi-fi prototype krever en webserver med PHP støtte. For utvikling og testing brukte vi \textit{Apache} web server.
Rapporten er skrevet som \LaTeX{} kode og er kompilert til PDF.
Kildekode for siste prototype og rapport er begge åpne og tilgjengelige via \textit{GitHub}, linker finnes i avsnitt \ref{sec:linker}.

\section{Linker og ressurser} \label{sec:linker}
Følgende liste består av klikkbare linker for alle prosjektets ressurser som presentasjoner, kildekode eller andre filer. 
\begin{itemize}
\item Pecha Kucha presentasjon for første delen av prosjektet\\ \url{https://prezi.com/z_1ipfrf4m_3/skolelinux-pa-ny-mate/}
\item Kildekode for prosjektet\\
\url{https://github.com/CervecerosCodigo/EasyDev}
\item Kilde for rapporten\\
\url{https://github.com/CervecerosCodigo/EasyDevRapport}
\item Forhåndsvisning av hi-fi prototype (klikk <<Logg inn>> uten brukernavn eller passord)
\url{http://student.cs.hioa.no/~s198569/EasyDev/login.php}
\item Prosjektposter
\end{itemize}