\chapter{Introduksjon}

\section{Om rapporten}
Rapporten består av flere kapitler som kan leses hver for seg og som har hvert sine formål. Det er lagt stor vekt på dokumentasjon av beslutninger, og prosessen underveis.



\begin{description}

\item[Introduksjonen] Går gjennom litt av forutsetningene for oppgaven, målene vi har satt oss, tolkningen av oppgaven og valgene vi har tatt på bakgrunn av det. 

\item[Hva er EasyDev?] En introduksjon til produktet. gir kort beskrivelse av målgruppe og hvordan det er tenkt at produktet skal brukes.

\item[Prosessdokumentasjon] Her beskrives det hvordan vi har jobbet med utviling av våre prototyper. Hvilke valg vi har tatt og hvorfor vi har tenkt som vi har gjort.

\item[Brukertesting] Vi har brukertestet vår hi-fi prototype på noen relevante brukere og gir en kort oppsumering av resultatene her.

\item[Evaluering] Her ser vi med et kritisk øye på arbeidet vårt. Vi diskuterer selve produktet, begrensninger og hva vi kunne ha gjort annerledes hvis...

\end{description}



\section{Tolkning av oppgaven}
Oppgaven skulle egentlig besvare følgende spørsmål: 
\begin{quote}
Et	blikk	inn	i	framtida:	Hvordan	kan	vi	om	10-20	år	bruke	teknologi	
– både	dagens	og	tenkt	framtidig	– til	å	gjøre	livet	bedre	for	
mennesker	som	har	spesielle behov.
\end{quote}

Vi tolkede oppgave litt anerldes. Som studenter i adre årskurs på høgskolen vet vel og godt hvordan det er å begynne på utdanningen og blir direkte kastet inn i en verden av teksbaserte konfigurasjonfiler. Vi ville derfro lage noe som ikke finnes fra før men kan hjelpe studenter med å starte opp de viktige delene i utdanningen sin insteden for å bruke tid på å sette opp et fungerende utviklingsssytem. 

\section{Formål}
Å lage et system for nye studenter som underletter daglig arbeid og tillater å sette igang med utvikling og gruppearbeid uten å bruke tid på å studere manualer og konfigurajon av system.

\section{Mål}
Følgende mål ble satt opp for å gjøre det enkelt å avgrense oppgaven:
\begin{description}
\item[Teknologi] definer vhilken tekngologi som vi ønsker at vår produkt skal basere seg på.
\item[Målgruppe] er det noen målgruppe som vi spesifikt ønsker å jobbe mot.
\item[Fremtid] Er løsningen fremtidsorienter. Finns en andre produkter som gir samme funksjonalitet som kan ta over i fremtiden?
\item[Faglig utfordring] Strekker vi oss lagt nok? Dersom vi ønsker å faktisk lage et produkt som idéen vår skal arbeidet også bidra til at vi blir flinkere utvilkere under reisen.

\end{description}

\section{Avgrensninger}
\begin{description}
\item[Hva oppgave ikke skal gjøre]
Oppgaven skal ikke resultere i et ferdi produkt. Dette er ikke mål med kurset og vi har ikke tilgjengelig tid for å ta frem både akritektur og kode for en fungerende produkt og løsning.
\item[Hva oppgaven skal levere]
Oppgaven skal resultere i en low-fi og en hi-fi prototype for å visualisere hvordan vi som gruppe har tenkt at vårt produkt skal fungere. Prototypen skal gi en <<\textit{technology preview}>> over noen funksjoner som er tenkt at skal implementeres.
\end{description}

\section{Hjelpemidler og verktøy} \label{sec:hjelpogverkt}
Til samarbeid i gruppen var brukt Google Drive og GitHub for utveksling av kildekode. Mockups til første protoype ble oprettet via \href{https://balsamiq.com/}{balsamiq mockups} plugin til Google Drive. \textit{Pecha Kucha} presentasjon er opprettet med \href{https://prezi.com/z_1ipfrf4m_3/skolelinux-pa-ny-mate/}{prezi}. Viktisgste UI elemter til hi-prototypen er lagd med \href{http://getbootstrap.com/}{bootstrap} og utformet med hjelp av med \href{http://designmodo.github.io/Flat-UI/}{FlatUI}. Til utvikling av kildekode bruktes Eclipse IDE med JavaScript, Web, og PHP eclipse miljø. Hi-fi prototype krever en webserver med PHP støtte. for utvikling og testing bruktes Apache web server. Kildekode er tilgjengelig på prosjektets git repository
\href{https://github.com/CervecerosCodigo/EasyDev}{https://github.com/CervecerosCodigo/EasyDev}.
Rapporten er skrevet som \LaTeX{} kode og er tilgjengelig via en separat git repository \\ \href{https://github.com/CervecerosCodigo/EasyDevRapport}{https://github.com/CervecerosCodigo/EasyDevRapport}

