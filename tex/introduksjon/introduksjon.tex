\chapter{Introduksjon}

\section{Om rapporten}
Rapporten består av flere kapitler som kan leses hver for seg og som har hvert sine formål. Det er lagt stor vekt på dokumentasjon av beslutninger, og prosessen underveis. 
De viktigste punktene i rapporten er de som omhandler tolkning av oppgaven (avsnitt \ref{sec:tolkning}), avgrensningene vi gjorde (avsnitt \ref{sec:avgrensninger} og \ref{sec:begrensninger}), hvorfor vi har gjort som vi har gjort osv. 

\begin{description}

\item[Introduksjonen] Går gjennom litt av forutsetningene for oppgaven, målene vi har satt oss, tolkningen av oppgaven og valgene vi tok på bakgrunn av det. 

\item[Hva er EasyDev?] En introduksjon til produktet. gir kort beskrivelse av målgruppe og hvordan det er tenkt at produktet skal brukes.

\item[Prosessdokumentasjon] Her beskrives det hvordan vi har jobbet med utvikling av våre prototyper. Hvilke valg vi har tatt underveis og hvorfor vi har tenkt som vi har gjort.

\item[Brukertesting] Vi har brukertestet vår hi-fi prototype på noen relevante brukere og gir en kort oppsummering av resultatene her.

\item[Evaluering] Her ser vi med et kritisk øye på arbeidet vårt. Vi diskuterer selve produktet, begrensninger og hva vi kunne ha gjort annerledes hvis...

\end{description}



\section{Tolkning av oppgaven} \label{sec:tolkning}
Oppgaven skulle egentlig besvare følgende spørsmål: 
\begin{quote}
\textit{Et	blikk	inn	i	framtida:	Hvordan	kan	vi	om	10-20	år	bruke	teknologi	
– både	dagens	og	tenkt	framtidig	– til	å	gjøre	livet	bedre	for	
mennesker	som	har	spesielle behov.}
\end{quote}

Da vi fikk tildelt prosjektoppgaven hadde vi allerede tenkt mye på et interessant prosjekt vi kunne jobbe med og som vi mente passet inn i MMI-faget. Da vi ble tildelt oppgavene som ga rammene for prosjektet var de en del annerledes enn det vi hadde lagt opp til.\\

Som studenter i det andre året på høgskolen vet vi veldig godt hvordan det er å begynne på utdanningen og det å bli kastet inn i en verden av nye verktøy, konfigurasjon av egne maskiner og lignende for å få tilgang til skolens ressurser. Vi ville derfor lage noe som ikke finnes fra før men kan hjelpe studenter med å komme i gang med pensum i stedet for å bruke tid på å sette opp et fungerende utviklingsmiljø.

Vi mener derfor at en portal for alle IT-studenter ved høgskolen som tilrettelegger for økt læring og gir bedre verktøy for samarbeid og gruppearbeid er et fullgodt alternativ til den oppgaveteksten som vi ble tildelt. Det ble selvsagt avklart med lærer før vi satt i gang med arbeidet. 

\section{Formål}
Å lage et system for nye studenter som forenkler arbeidet og mulighetene for å sette igang med utvikling og å jobbe sammen i gruppearbeid uten å bruke tid på å studere manualer og konfigurasjon av systemet. Dagens løsning fungerer dårlig da en ikke kan konfigurere miljøet man jobber mot, og ikke har mulighet for å jobbe sammen i gruppe på én brukers system. Les mer om det i avsnitt \ref{sec:hvaereasydev}.

\section{Mål}
%Gi et nytt innspill på hvordan en institusjon som HIOA kan organisere og implementere verktøyene de tilbyr IT-studenter i årene fremover. 
%Målet er videre at det vi har kommet frem til i prosjektet skal kunne realiseres om en har tid og ressurser. Det er da følgelig ingen sci-fi løsning der man er avhengig av fremtidig utvikling på andre plan for å kunne implementere vårt produkt.
\begin{itemize}
\setlength{\itemsep}{1pt}
\setlength{\parskip}{0pt}
\setlength{\parsep}{0pt}
\item Ta frem produktidéer. Velg en av disse idéer til å lage utkast.
\item Ta frem mockups (low-fi protitype) som viser systemet basert på den valgte idéen.
\item Videreutvikle alle tilgjengelige mockups av skjermbilder til en hi-fi prototype som representerer første utkast over systemets grensesnitt.
\end{itemize}

\section{Avgrensninger} \label{sec:avgrensninger}

\subsection{Hva skal prosjektet ikke levere}
Prosjektet skal ikke resultere i et ferdig produkt. Det er ikke et mål med kurset eller prosjektoppgaven. Grunnet tiden har vi heller ikke mulighet til å implementere arkitektur og utvikle et fungerende produkt.

\subsection{Hva skal prosjektet levere}
Prosjektet skal resultere i en lo-fi og en hi-fi prototype for å visualisere hvordan vi som gruppe har tenkt at vårt produkt skal fungere. Prototypen skal gi en <<\textit{technology preview}>> over noen funksjoner som er tenkt at skal implementeres.
Hovedfokuset blir likevel lagt på rapporten ettersom det er der man finner informasjon om gruppens visjoner, hvilke valg som blir tatt, prosessen underveis og vår egen evaluering av produktet. I tillegg lages det en poster som skal presentere løsningen vår.

\subsection{Hvordan er prosjektet avgrenset}
Følgende avgrensninger ble satt opp for å gjøre det enklere å avgrense oppgaven:
\begin{description}
\item[Tid] Tilgjengelig tid gir automatisk avgrensning i forhold til hvor konkret implementasjonen kan bli.
\item[Teknologi] Definere hvilken teknologi vi ønsker at vårt produkt skal basere seg på.
\item[Målgruppe] Hvilken målgruppe ønsker vi å jobbe mot.
\item[Fremtid] Er løsningen fremtidsorientert? Finnes en andre produkter som gir samme funksjonalitet?
\item[Faglig utfordring] Strekker vi oss lagt nok, eller for langt?

\end{description}

\section{Hjelpemidler og verktøy} \label{sec:hjelpogverkt}
Underveis i prosjektet har vi brukt \textit{Google Drive} og \textit{GitHub} for utveksling av kildekode. Mockups til første prototype ble opprettet via \href{https://balsamiq.com/}{\textit{balsamiq mockups}} plugin til \textit{Google Drive}. \textit{Pecha Kucha} presentasjon er opprettet med \href{https://prezi.com/z_1ipfrf4m_3/skolelinux-pa-ny-mate/}{\textit{Prezi}}. Viktigste UI elemeter til hi-prototypen er laget med \href{http://getbootstrap.com/}{bootstrap} og utformet med hjelp av med \href{http://designmodo.github.io/Flat-UI/}{FlatUI}. Til utvikling av kildekode brukte vi \textit{Eclipse IDE} med \textit{JavaScript}, \textit{Web}, og \textit{PHP} eclipse miljø. Hi-fi prototype krever en webserver med PHP støtte. For utvikling og testing brukte vi \textit{Apache} web server.
Rapporten er skreven som \LaTeX{} kode og er kompilert til PDF.
Kildekode for seineste prototype og rapport er begge åpent og tilgjengelige via \textit{GitHub}, linker finnes i avsnitt \ref{sec:linker}.

\section{Linker og resurser} \label{sec:linker}
Følgende liste består av klikkbare linker for alle prosjektets resurser som presentasjoner, kildekode eller andre filer. 
\begin{itemize}
\item Pecha Kucha presentasjon for første delen av prosjektet\\ \url{https://prezi.com/z_1ipfrf4m_3/skolelinux-pa-ny-mate/}
\item Kildekode for prosjektet\\
\url{https://github.com/CervecerosCodigo/EasyDev}
\item Kilde for rapporten\\
\url{https://github.com/CervecerosCodigo/EasyDevRapport}
\item Forhåndsvisning av hi-fi prototype (klikk <<Logg inn>> uten brukernavn eller passord)
\url{http://student.cs.hioa.no/~s198569/EasyDev/login.php}
\item Prosjektposter
\end{itemize}