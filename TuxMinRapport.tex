\documentclass[a4paper,11pt]{report}

%TexLive pakker
\usepackage[utf8]{inputenc}
\usepackage[T1]{fontenc}
\usepackage{lmodern}
\usepackage[norsk]{babel}
\usepackage{parskip}
\usepackage{graphicx}
\usepackage{caption}
\usepackage{subcaption}
\usepackage{titlepic}
\usepackage{a4wide}
\usepackage{lettrine}
%\usepackage[htt]{hyphenat}
\usepackage{enumitem}
\usepackage{color}
\usepackage{hyperref}
\usepackage{listings}
\usepackage[section]{placeins}
% xcolor og fix-cm brukes til fremsiden
\usepackage{xcolor} 
\usepackage{fix-cm}
\usepackage{hyperref}
%\usepackage{url}
\usepackage{colortbl}
\usepackage{tabularx}

% Definerer litt farger for en fin tabell som skal presentere fltUI
% Kilde: http://flatuicolors.com/
\definecolor{Turquoise}{RGB}{26, 188, 156}
\definecolor{Emerald}{RGB}{46, 204, 113}
\definecolor{PeterRiver}{RGB}{52, 152, 219}
\definecolor{Amethyst}{RGB}{155, 89, 182}
\definecolor{WetAsphalt}{RGB}{52, 73, 94}

\definecolor{GreenSea}{RGB}{22, 160, 133}
\definecolor{Nephritis}{RGB}{39, 174, 96}
\definecolor{BelizeHole}{RGB}{41, 128, 185}
\definecolor{Wisteria}{RGB}{142, 68, 173}
\definecolor{MidnightBlue}{RGB}{44, 62, 80}

\definecolor{SunFlower}{RGB}{241, 196, 15}
\definecolor{Carrot}{RGB}{230, 126, 34}
\definecolor{Alizarin}{RGB}{231, 76, 60}
\definecolor{Clouds}{RGB}{236, 240, 241}
\definecolor{Concrete}{RGB}{149, 165, 166}

\definecolor{Orange}{RGB}{243, 156, 18}
\definecolor{Pumpkin}{RGB}{211, 84, 0}
\definecolor{Pomegranate}{RGB}{192, 57, 43}
\definecolor{Silver}{RGB}{189, 195, 199}
\definecolor{Asbestos}{RGB}{127, 140, 141}
 
\lstdefinestyle{java1}{
  language=Java,
  basicstyle={\ttfamily\footnotesize},
  numberstyle={\ttfamily\footnotesize},
  keywordstyle={\ttfamily\footnotesize\color[rgb]{0,0,1}},
  commentstyle={\ttfamily\footnotesize\color[rgb]{0.133,0.545,0.133}},
  stringstyle={\ttfamily\footnotesize\color[rgb]{0.627,0.126,0.941}},
  breaklines=true,
  columns=fixed,
  extendedchars=true,
  numbers=left,
  numbersep=3pt,
  showspaces=false,
  showstringspaces=false,
  stepnumber=1,
  tabsize=2
}
\lstset{style=java1,
literate=%
{æ}{{\ae}}1
{å}{{\aa}}1
{ø}{{\o}}1
{Æ}{{\AE}}1
{Å}{{\AA}}1
{Ø}{{\O}}1
}
\renewcommand\lstlistingname{Eksempel}
\renewcommand\lstlistlistingname{Eksempler}

\usepackage[Bjornstrup]{fncychap}

\usepackage{fancyhdr}
\setlength{\headheight}{15.2pt}
\pagestyle{fancy}
\fancypagestyle{plain}{ %
  \fancyhf{} % remove everything
  \renewcommand{\headrulewidth}{0pt} % remove lines as well
  \renewcommand{\footrulewidth}{0pt}
}

%Her kommer opsett av info for pdf filen 
% \pdfinfo{%
%  /Title    (Rapport prosjektoppgave i programutvikling)
%  /Author   ()
%  /Creator  ()
%  /Producer ()
%  /Subject  ()
%  /Keywords ()
% }



\begin{document}
\begin{titlepage}
%\begin{center}
 



\title{
\rule{\linewidth}{0.5mm}
\textsc{\LARGE EasyDev}
\rule{\linewidth}{0.5mm}
}


\titlepic{\includegraphics[width=50mm]{./img/fremside/logo.png}}



\author{\textsc{\Large Gruppe TuxMin}\\ \hline \\ Espen Zaal s198599\\Lukas
Larsed s198569\\Petter Knagenhjelm Lysne s198579}



\date{\today}



%\end{center}
\end{titlepage}


% Her kommer sammendrag som skal være det som kommer først etter fremsiden.  
\begin{abstract}
\lettrine[lines=3]{E}{asyDev} er verktøy enhver IT-student burde ha tilgang til. Enkelt tilgjengelig via en nettleser gir mulighet til enkel administrasjon av webområder, SQL-databaser og deleing av filer mellom gruppemedlemmer. Løsningen er basert på \textit{Linux}, \textit{Apache} web server og \textit{MySQL} database der hver student får tildelt en virtuell maskin. 
I den virtuelle maskinen kan en opprette, endre og slette databaser samt installere nødvendige moduler for arbeidet sitt. 
Produktet har et utvalg av nettbaserte editorer for \textit{SQL}-spørringer, \textit{HTML} og \textit{Javascript}. 
Alt dette er satt sammen i et system med tiltalende grensesnitt som gjør at studenter raskt kan komme i gang med utvikling og skolearbeid.
Rapporten beskriver prosess og besluttinger som var tatt underveis  og ligger til grunn for utforming av en prototype for EasyDev. 
I tillegg var det også gjennomført brukertester for prototypen.
Resultat fra brukertestene gav en indikasjon på at både idé og utforming gir en god balanse mellom brukeropplevelse og funksjon som resulterer i et balansert produkt etterspurt av studentene.



\end{abstract}
\tableofcontents
%\lstlistoflistings
\listoffigures
%\listoftables

%Her kommer alle linker til våre kapitel

\chapter{Introduksjon}

\section{Om rapporten}
Denne rapporten består av flere kapitler som kan leses hver for seg og som har hvert sine formål.


\begin{description}

\item[Introduksjonen] vil gå gjennom litt av forutsetningene for oppgaven, målene vi har satt oss, tolkningen av oppgaven og valgene vi har tatt på bakgrunn av det. 

\item[Ha er EasyDev?] en introduksjon til produktet. gir kort beskrivelse av målgruppe og hvordan det er tenkt at produktet skal brukes.

\item[Prosessdokumentasjon] her beskriver vi hvordan vi har gått til veis under utviling av våre prototyper. Hvilke beslutt vi har tatt og hvorfor vi har tenkt som vi har gjort.

\item[Brukertesting] vi har brukertestet vår hi-fi prototype på noen tenkbare brukere og gir en kort oppsumering av resultatene her.

\item[Evaluering] her se vir med et kritisk øye på arbeidet vårt. Vi diskuterer selve produktet, begrensninger og hva kunde vi ha gjort annerledes hvis...

\end{description}

\section{Formål}
Formål med oppgaven er å ta frem en idé og materisere den i en prototype som kan vise ekesmpel på hvordan slik teknologi kan i fremtiden gjøre livet bedre og enklere for mensker med spesielle behov. 

\section{Tolkning av oppgaven}
Oppgaven skulle egentlig besvare følgende spørsmål: 
\begin{quote}
Et	blikk	inn	i	framtida:	Hvordan	kan	vi	om	10-20	år	bruke	teknologi	
– både	dagens	og	tenkt	framtidig	– til	å	gjøre	livet	bedre	for	
mennesker	som	har	spesielle behov.
\end{quote}

Vi tolkede oppgave litt anerldes. Som studenter i adre årskurs på høgskolen vet vel og godt hvordan det er å begynne på utdanningen og blir direkte kastet inn i en verden av teksbaserte konfigurasjonfiler. Vi ville derfro lage noe som ikke finnes fra før men kan hjelpe studenter med å starte opp de viktige delene i utdanningen sin insteden for å bruke tid på å sette opp et fungerende utviklingsssytem. 

\section{Mål}
Følgende mål ble satt opp for å gjøre det enkelt å avgrense oppgaven:
\begin{description}
\item[Teknologi] definer vhilken tekngologi som vi ønsker at vår produkt skal basere seg på.
\item[Målgruppe] er det noen målgruppe som vi spesifikt ønsker å jobbe mot.
\item[Fremtid] Er løsningen fremtidsorienter. Finns en andre produkter som gir samme funksjonalitet som kan ta over i fremtiden?
\item[Faglig utfordring] Strekker vi oss lagt nok? Dersom vi ønsker å faktisk lage et produkt som idéen vår skal arbeidet også bidra til at vi blir flinkere utvilkere under reisen.

\end{description}

\section{Avgrensninger}
\begin{description}
\item[Hva oppgave ikke skal gjøre]
Oppgaven skal ikke resultere i et ferdi produkt. Dette er ikke mål med kurset og vi har ikke tilgjengelig tid for å ta frem både akritektur og kode for en fungerende produkt og løsning.
\item[Hva oppgaven skal levere]
Oppgaven skal resultere i en low-fi og en hi-fi prototype for å visualisere hvordan vi som gruppe har tenkt at vårt produkt skal fungere. Prototypen skal gi en <<\textit{technology preview}>> over noen funksjoner som er tenkt at skal implementeres.
\end{description}

\section{Hjelpemidler og verktøy}
Til samarbeid i gruppen var brukt Google Drive og GitHub for utveksling av kildekode. Mockups til første protoype ble oprettet via \href{https://balsamiq.com/}{balsamiq mockups} plugin til Google Drive. \textit{Pecha Kucha} presentasjon er opprettet med \href{https://prezi.com/z_1ipfrf4m_3/skolelinux-pa-ny-mate/}{prezi}. Viktisgste UI elemter til hi-prototypen er lagd med \href{http://getbootstrap.com/}{bootstrap} og utformet med hjelp av med \href{http://designmodo.github.io/Flat-UI/}{FlatUI}. Til utvikling av kildekode bruktes Eclipse IDE med JavaScript, Web, og PHP eclipse miljø. Hi-fi prototype krever en webserver med PHP støtte. for utvikling og testing bruktes Apache web server. Kildekode er tilgjengelig på prosjektets git repository
\href{https://github.com/CervecerosCodigo/EasyDev}{https://github.com/CervecerosCodigo/EasyDev}.
Rapporten er skrevet som \LaTeX{} kode og er tilgjengelig via en separat git repository \\ \href{https://github.com/CervecerosCodigo/EasyDevRapport}{https://github.com/CervecerosCodigo/EasyDevRapport}


\chapter{Hva er EasyDev?} \label{sec:hvaereasydev}

\lettrine[lines=2]{H}{} va er egentlig EasyDev? For å kunne fortelle hva EasyDev er må man gå tilbake et år i tid. Alle medlemmene i gruppen begynte sin utdanning på Informasjonsteknologi høsten 2013. Som relativt nye studenter skulle man begynne å utvikle websider og PHP-skript for websider. For å komme igang med slik utvikling må man kunne teste sin kode på en webserver hvilket er den eneste løsningen for å teste PHP-kode. 
Man har to muligheter: (1) kjøre koden på skolens webserver (alle studenter har sin egen bruker og filer som skal publiseres på nett legges i www-mappe) eller (2) sette opp en webserver på sin egen maskin.

Dersom man velger alternativ én krever det at man jobber direkte mot filene som finnes på serveren eller at man for hver gang som koden skal testes laster opp en ny versjon av filene til serveren. Det første alternativet er tregt og upraktisk og hvis man skal jobbe hjemmefra må det bruke VPN-tilkoblig, som i seg selv fungerer meget bra, men man savner mulighet for å samarbeide over slik løsning.

På grunn av disse praktiske begrensingene velger de fleste studentene å sette opp en server på sin egen maskin. Dersom man velger et slikt alternativ har man to valg: sette opp sin egen \textit{Apache} web-server med PHP-funksjonalitet eller gå for den enkleste og vanligste løsningen der man installerer en klone av Apache i form av XAMPP-applikasjon med et ferdig oppsett av PHP og webserver. Å sette opp sitt eget \textit{Apache} servermiljø er ikke noe som de fleste ønsker å sette seg inn i fordi det krever en del forkunnskaper og tid for å få serveren til å fungere som man ønsker. Dersom man skal bruke XAMPP gjøres det på en forhåndskonfigurert måte og servermiljøet kan ikke konfigureres fritt. XAMPP tillater ikke bruk av mange populære Apache plug-ins som mange av studentene ønsker ta i bruk etter som man utvikler ferdighetene i PHP og lignende teknologier.

Et annet aspekt er i forbindelse med gruppearbeid. I gruppearbeid skal flere personer samarbeide på ett sett av filer og dokument. Hvordan gjør man det? Den raskeste måten, som de fleste studenter velger, er at man setter opp en delt mappe mellom alle gruppemedlemmene på en av de mest populære synkroniseringstjenestene som \textit{Dropbox, Box, SkyDrive} eller \textit{Google Drive}. Det man raskt oppdager (oftest med negativt fortegn) er at man ganske så omgående vil bli offer for lagringskonflikter. Hvis to personer har åpnet en og samme fil samtidig kommer synkroniseringstjenesten til å opprette to versjoner av filen på grunn av lagringskonflikt. 
Men det er ikke alltid nødvendig at filen skal være åpen av to personer samtidig for å forårsake én konflikt. For eksempel, om ikke en endring blir synkronisert etter at en person er ferdig med filen og neste person begynner å endre samme fil før synkroniseringen er fullført så blir den første personens endringer borte. 

Synkroniseringstjenester er tross alt et meget bra verktøy så lenge flere brukere ikke skal editere de samme filene samtidig. Det eneste alternativet som finnes er å starte bruk av et versjonskontrollsystem som f.eks. git, cvs eller subversion. Versjonskontrollsystem er et fantastiske verktøy men det alle disse systemene har til felles er at de har en meget bratt læringskurve. Dette er noe som ofte skremmer nye brukere og det er vanskelig å motivere alle gruppens medlemmer til å faktisk bruke tid på å lære seg et nytt komplekst verktøy som i tillegg må brukes via terminal\footnote{Det finnes mange brukergrensesnitt spesielt for git, men disse viser seg ofte å være vanskeligere å bruke enn den terminalbaserte løsningen. Bruk av versjonshåndtering via terminal gir garantert best kontroll og brukeropplevelse.}.

Andre utfordringer oppstår da man for første gang har behov å sette opp en egen database. I skolen er det vanlig at man jobber med \textit{MySQL} databasesystem. Det man får tilbudt er ferdig konfigurert, der man ikke har muligheter til å slå på tilleggsfunksjoner man gjerne trenger å bruke i sitt arbeid. Det er også vanlig å bruke \href{http://en.wikipedia.org/wiki/PhpMyAdmin}{\textit{phpMyAdmin}} som er et webbasert grensesnitt til konfigurasjon av en \textit{MySQL} server. Grensesnittet krever dog at man først skal ha konfigurert både webserver og php for at dette skal fungere.

Dette er kun to eksempel på frustrasjoner som møter en ny student på en IT-utdanning. Det er ikke bare pensum og lesestoff men man må også lære seg mestre flere verktøy som er nødvendige for å komme igang med både gruppearbeid og sin egen utvikling. Så hvis en ser tilbake på det spørsmålet som egentlig ble stilt i begynnelsen av dette kapittelet: Hva er EasyDev? Da kan det egentlig besvares ganske enkelt: 

\begin{quotation}
\emph{\textbf{EasyDev} er et brukergrensesnitt til en virtuell Linux maskin som inkluderer alle verktøy og moduler som en student på en IT-utdanning har behov for. Derfor, EasyDev er et system som skal gjøre det mulig å komme igang med skolearbeid og utvikling uten å lese en linje av kjedelig dokumentasjon og istedet fokusere på å skape.}
\end{quotation}


\section{Målgruppe} \label{sec:målgruppe}
Innledningsvis begrenses målgruppen til nye IT studenter ved universitet og høgskoler. Ettersom systemet skal følge Linux programmeringsstil\cite{book:unixprog} vil det være mulig å skalere det etter egne behov. Det kan på et senere tidspunkt bli aktuelt å sette opp et system som kun er rettet mot andre utdanninger, f.eks. kun med støtte for meldingssystem og gruppearbeid.

\section{Eksisterende løsninger}
Undersøkelsene som ble gjort i forbindelse med prosjektet fant ikke noe system som gir mulighet for akkurat den funksjonaliteten som denne løsningen tilbyr. Det finnes system som \href{http://en.wikipedia.org/wiki/Webmin}{\textit{webmin}} men det er mer rettet mot systemadministratorer og tilbyr ingen <<desktop>>-funksjonalitet. Dette er en skreddersydd løsning som til å begynne med er rettet mot kun IT-studenter, og gir samarbeidsløsninger som vi ikke har funnet andre steder i markedet i dag.

\section{Nyskapende}
Det er flere aspekter man vil legge merke til som dette prosjektet tilbyr og som er helt nyskapende og som det ikke er funnet i noe liknende system av denne typen. Listen over funksjonalitet er ganske lang, og derfor vil rapporten kun beskrive de viktigste delene. For en utfyllende beskrivelse av alle de modulene som er kartlagt og planlagt, henvises leser til appendiks \ref{app:funksjonalitet}, side \pageref{app:funksjonalitet}.


\begin{description}
\item[Modulbasert] Ingen av modulene som brukes til konfigurering av systemet er hardkodet i EasyDev. Alt er modulbasert, det medfører at man kan enkelt legge til eller fjerne pakker etter eget ønske. Hver skole eller avdeling som tar i bruk EasyDev kan selv velge hvilke pakker og moduler som skal tas i bruk. Det gir også mulighet for utvikling av egne moduler. Brukere som ønsker spesifikk funksjonalitet kan utvikle egne moduler mot EasyDev sitt API. 
\item[Laget for bruk i skolen] Det at man kan få et ferdigdefinert system, tilpasset av hver skole, gjør at det er veldig fleksibelt fra skolen sitt synspunkt, samtidig gir det hver enkelt student mer råderett over det systemet han/hun har fått tildelt. Det at man samtidig kan invitere medstudenter til å jobbe på sitt eget system gjør at gruppearbeid forenkles. Det er ment at EasyDev skal gi den beste mulige løsningen for alle parter som er involvert.
\item[Enkel videreutvikling]
Ettersom løsningen bygger på at man har et virtuelt operativsystem som backend kan den egentlig videreutvikles etter eget ønske. Et eksempel på dette kan være at man tilpasseer den til å kjøre på en meget rimelig hardware som \textit{Rasberry-Pi} og brukes på billige skolemaskiner i utviklingsland.
\end{description}
\chapter{Prosessdokumentasjon}
%\marginpar{
%Test for sitering av kilder.\cite{forelesning:tulpesh}\cite{book:utforming}\cite{book:desintsystems}
%}
\lettrine[lines=2]{F}{} ølgende kapittel beskriver hvordan hele utviklingsprosessen av prototypene har foregått og som illustrerer hele prosessen fra idé til mockup til hi-hi prototype. Kapittelet består av mange bilder for å på enklest mulig måte illustrere prosessen. For eventuell beskrivelse av funksjonalitet av de modulene som er synlige på bildene henvises det til appendiks \ref{app:funksjonalitet}.




\section{Første utkast} \label{sec:utkast}
\marginpar{Viktige prinsipper:
feedback, constraints (bruker får ikke gjøre feil), affordances
}
\marginpar{Husk at vi borde ta med en navigasjonkart}
\marginpar{Støtteord:
Usability (brukbarhet): konsistens, brukerkontroll, passende presentasjon.
}
Etter at idéen om hva vi skulle jobbe med var klar tok det ikke lang tid før vi ble enige om hvordan vi skal sette sammen vårt forslag til en helhet. Hele gruppen var tydelig på at vi i stor grad ønsket å benytte oss av en løsning som følger Gestalt-prinsipper for god design og layout.
Systemet skulle kunne administreres via nettleser som gjør det mulig å bruke systemet fra hvilken som helst maskin og med eventuelle tilpasninger kan også gjøres slik at man også kan jobbe fra nettbrett eller telefon.
På det tidspunktet handlet diskusjonen om hvordan vi skulle fordele de forskjellige modulene og funksjonalitetene på så få hovedområder som mulig, der hvert av de områdene blir et element i hovedmenyen. 
I det første utkastet bestod disse av:
%\begin{center}
%\texttt{Editorer | Meldinger | Resurser | System | Servere}
%\end{center}
\begin{table}[h]
\newcommand{\paddA}{1ex}
\newcommand{\paddB}{0.2ex}
\begin{tabularx}{\textwidth}{|*5{>{\centering\arraybackslash}X|}@{}|}
\hline
\vspace*{\paddA} Editorer & \vspace*{\paddA} Meldinger & \vspace*{\paddA} Resurser & \vspace*{\paddA} System & \vspace*{\paddA} Servere \\[2ex] 
\hline
\vspace*{\paddB} Java		&	\vspace*{\paddB} HTML Val			& 	\vspace*{\paddB} Meldinger 	& 	\vspace*{\paddB} Apache 	& \vspace*{\paddB} Brannmur \\
PHP		& 	CSS Validering 	& 				& 	MySQL 	& Brukergrupper \\
MySQL	& 	W3Schools 		& 				& 	SSH 		& Logg \\
C 		& 	DropBox 			& 				& 			& Filbehnadling \\
 		& 	Terminal \vspace*{\paddB} 		& 				& 			& Systeminfo \vspace*{\paddB} \\

\hline
\end{tabularx} 
\end{table}


\begin{figure}
%bruk \begin{figure}[ht] dersom figuren ikke skal flyte
\includegraphics[width=\textwidth,height=\textheight,keepaspectratio]{./img/prosessdokumentasjon/foersteutkast/foerste.jpg}
\caption[Første utkast]{Første utkast over brukergrensesnittet.}
\label{fig:foersteutkast}
\end{figure}

\section{Low-fi prototype} \label{sec:lowfi}
\emph{Avsnittet inneholder et stort antall mockups som representerer forskjellige skjermbilder. Det er flere bilder som er plassert i teksten, og dersom noen av komponentene i bildene oppleves for små er samtlige av bildene presentert i høy oppløsning i appendiks \ref{sec:appendiksLowFi} side \pageref{sec:appendiksLowFi}.}

Det er ganske vanlig at man til en low-fi prototype bruker papp eller papir for å visualisere hvordan man kan bruke en GUI. Vi valgte å benytte oss av \href{http://balsamiq.com/products/mockups/}{balsamiq mockups} som ga oss mulighet til å ikke bare visualisere hvordan brukergrensesnittet skulle se ut men også legge inn enkel funksjonalitet. Blant annet at man fikk mulighet til å klikke seg videre til neste skjermbilde direkte ved å trykke på fra menyer eller knapper i mockupen. Dette ga en ganske grei opplevelse av hvordan det kommer til å føles når man bruker produktet.


Dersom vi tar et rask overblikk over framsiden i figur \ref{fig:lowfi_fremside} ser vi at hver enkelt modul er gruppert etter gestalt prinsipper for proksimitet og elementer.\cite{forelesning:tulpesh}

Prototypen inkluderer eksempel på konfigurasjon av to forskjellige funksjoner.
Det er vanlig at i en prototype fokuserer man på en horisontal eller vertikal implementasjon. Der man i en horisontal implementasjon implementerer få eller ingen funksjoner som går i dybden med istedet forsøker man å vise bredden av muligheter i et system eller grensesnitt. I den vertikale implementasjonen velger man istedet å begrense bredden på hva prototypen kan vise eller gjøre og istedet implementerer en eller flere funksjoner i dybden.\cite{book:utforming}
Vi valgte å avgrense til implementasjon av Apache webserver og oppsett av brukere og grupper. 

\begin{figure}
%bruk \begin{figure}[ht] dersom figuren ikke skal flyte
\includegraphics[width=\textwidth,height=\textheight,keepaspectratio]{./img/prosessdokumentasjon/lowfi/fremside.png}
\caption[Low-fi prototype]{Fremside for EasyDev i første low-fi prototype.}
\label{fig:lowfi_fremside}
\end{figure}


I påfølgende avsnitt beskriver vi i detalj funksjonaliteten til to moduler. Det er grunnlegende moduler i systemet og som har vært med siden den opprinnelige idéen ble unnfanget. De to modulene er gode eksempler på hvordan vi har tenkt at produktet skal fungere.

\subsection{Oppsett av apache webserver}
Her går vi nærmere inn på hvordan det er tenkt at en submodul for konfigurasjon og oppsett av serverelement skal brukes. Alle trinn som beskrives er presentert i figur \ref{fig:lowfiapache}.

Vi utgår fra at det er intuitivt å velge modul \texttt{Servere} dersom man ønsker å konfigurere en tjener. Modulen ligger i hovedmenyen med en nedtrekksmeny som viser hvilke servere man har mulighet til å konfigurere, figur \ref{fig:apache1}. 
Apache webserver trenger å vite hvilke mapper som skal deles ut som webområder. Hvis man installerer apache på en Linux manskin kommer den som standard å lese fra mappe \texttt{/var/www} og som har kun root tilgang. Vi vil ikke endre rettigheter for denne mappe ettersom de vi stride mot standarden for rettigheter i systemet og utgjøre en sikkerhetsrisiko.\cite{book:unixprog}
Det vi ønsker å gjøre istedet er å definere om innstillinger i Apache slik at serveren leser fra brukerens sin egen mappe. Dette er normalt en utfordring ettersom det må gjøres i form av to til tre trinn:\footnote{Antall trinn her er ganske avhengig av hvilket Linux system som brukes}

\begin{itemize}
\item Endre konfigurasjon i apache.conf til å lese fra brukerens hjemmemappe. Det må opprettes en spesifikk mappe f.eks. \texttt{/home/bruker/html}. 
I noen tilfelder er det nødvendig å installere plugin \href{http://httpd.apache.org/docs/2.2/mod/mod_userdir.html}{\texttt{mod\_{}userdir}}. Dette vil gjøre det mulig å lese filene vis følgende adresse: \texttt{http://localhost/\~{}bruker/fil.html}

\item Endre rettigheter til den mappen i hjemmemappen som filene skal leses fra. Dette forutsetter at brukeren og Apache webserver er i samme brukergruppe. Oftest må brukeren bli lagt til \texttt{www} eller \texttt{httpd} gruppe på systemet. Det er selvfølgelig mulig å åpne mappen helt, slik at alle grupper i systemet vil ha tilgang til denne, men slik tilnærming senker sikkerheten i stor grad i et flerbrukersystem. Dette er noe som kan være veldig vanskelig for nybegynnere, der man i starten faktisk velger å åpne mappen helt ettersom man ofte ikke greier å teste om rettighetene er satt riktig.

\item Systemer som benytter seg av SELinux (\href{http://en.wikipedia.org/wiki/Security-Enhanced_Linux}{Security Enhanced Linux}) må i tillegg legge til en policy der Apache skal få lov til å lese fra brukermapper. Dette er på grunn at SELinux er et oppsett av rettigheter som i tillegg overstyrer vanlig skriv og leserettigheter i systemet på kernel nivå\footnote{Dette er ikke noe som vi skal beskrive i detalj og nevnes her kun for å illustrere problemstillingen som brukeren kan bli utsatt for.}. 
\end{itemize}

Eksemplene over viser hvor vanskelig det kan være å faktisk sette opp Apache webserver på sin egen maskin.  
Dette er et godt eksempel på hvor EasyDev kommer inn med enkle løsninger. Apache modulen i EasyDev er tiltenkt brukt slik at alle disse konfigurasjonene blir tatt hånd om av systemet, og brukeren har et enkelt grensesnitt å forholde seg til.

Dersom vi ser i figur \ref{fig:apache2} ser vi at i konfigurasjonen kan brukeren velge hvilke mapper som skal inkluderes som et webområde. Altså mapper som webserveren lese fra og vise som websider.
I neste trinn (figur \ref{fig:apache3}) blir brukeren presentert for hvilke av de valgte mappene man ønsker man å benytte for å aktivere <<directory browsing>>. Dette er en funksjon som gjør det mulig å se innholdet i en mappe direkte i nettleseren. Dette er normalt en funksjon som ikke brukes på en <<live>> server men er meget brukbart ved både utvikling og testing. 
Figur \ref{fig:apache4} viser hvilke typer plugins som kan velges som tillegg i Apache installasjonen. Disse kan for eksempel være plugins som PHP eller MySQL eller andre. Hensikten er at man kan samle alle slike tillegg på en plass slik at brukeren kan velge disse på et senere tidspunkt.
Siste eksempel (figur \ref{fig:apache5}) viser mulighet for konfigurasjon av noen av mer avanserte tilleggsmoduler. Disse kan f.eks. være støtte for flere vituelle servere av Apache noe som gir bedre skalerbarhet ved utvikling av flere prosjekter parallelt. 
% med flagg p setter vi hele denne figuren på en egen side
\begin{figure}[p]
        \centering
        \begin{subfigure}[b]{0.48\textwidth}
                \includegraphics[width=\textwidth]
                {./img/prosessdokumentasjon/lowfi/apache1.png}
                \caption{Fremside -> Sett opp webserver}
                \label{fig:apache1}
        \end{subfigure}%
        ~ %add desired spacing between images, e. g. ~, \quad, \qquad, \hfill etc.
          %(or a blank line to force the subfigure onto a new line)
        \begin{subfigure}[b]{0.48\textwidth}
                \includegraphics[width=\textwidth]
                {./img/prosessdokumentasjon/lowfi/apache2.png}
                \caption{Valg av mapper}
                \label{fig:apache2}
        \end{subfigure}
       
        \vspace{0.6cm}
        \begin{subfigure}[b]{0.48\textwidth}
                \includegraphics[width=\textwidth]
                {./img/prosessdokumentasjon/lowfi/apache3.png}
                \caption{Aktivere visning av innhold i mapper}
                \label{fig:apache3}
        \end{subfigure}
        \hspace{0.05cm}
        \begin{subfigure}[b]{0.48\textwidth}
                \includegraphics[width=\textwidth]
                {./img/prosessdokumentasjon/lowfi/apache4.png}
                \caption{Valg av moduler}
                \label{fig:apache4}
        \end{subfigure}
        
        \vspace{0.6cm}
        \begin{subfigure}[b]{0.48\textwidth}
                \includegraphics[width=\textwidth]
                {./img/prosessdokumentasjon/lowfi/apache5.png}
                \caption{Avanserte Apache moduler}
                \label{fig:apache5}
        \end{subfigure}
        \vspace{0.1cm}
        \caption[Konfigurasjon av Apache webserver]{Eksempel på konfigurasjon av Apache webserver.}\label{fig:lowfiapache}
\end{figure}


\subsection{Oppsett av brukere og grupper}
I følgende modul har man mulighet til å sette både brukergrupper, brukere og rettigheter en en enkelt bruker eller hele grupper av brukere. Modulen er tenkt å brukes i forbindelse med gruppearbeid slik at flere brukere kan dele på samme resurser. De eksempler som beskrives her kan betraktes i figur \ref{fig:lowfibrukere}. 
% med flagg p setter vi hele denne figuren på en egen side
\begin{figure}[h]
        \centering
        \begin{subfigure}[b]{0.48\textwidth}
                \includegraphics[width=\textwidth]
                {./img/prosessdokumentasjon/lowfi/b1.png}
                \caption{Fremside -> Velg brukeroppsett}
                \label{fig:brukere1}
        \end{subfigure}%
        ~ %add desired spacing between images, e. g. ~, \quad, \qquad, \hfill etc.
          %(or a blank line to force the subfigure onto a new line)
        \begin{subfigure}[b]{0.48\textwidth}
                \includegraphics[width=\textwidth]
                {./img/prosessdokumentasjon/lowfi/b2.png}
                \caption{Grupper}
                \label{fig:brukere2}
        \end{subfigure}
       
        \vspace{0.4cm}
        \begin{subfigure}[b]{0.48\textwidth}
                \includegraphics[width=\textwidth]
                {./img/prosessdokumentasjon/lowfi/b3.png}
                \caption{Brukere}
                \label{fig:brukere3}
        \end{subfigure}
        \hspace{0.02cm}
        \begin{subfigure}[b]{0.48\textwidth}
                \includegraphics[width=\textwidth]
                {./img/prosessdokumentasjon/lowfi/b4.png}
                \caption{Moduler og resurser for gruppe}
                \label{fig:brukere4}
        \end{subfigure}
        \vspace{0.1cm}
        \caption[Konfigurasjon av brukere og grupper]{Eksempel på konfigurasjon av brukere og grupper.}\label{fig:lowfibrukere}
\end{figure}
Vi begynner på samme måte som i forrige eksempel og starter på fremsiden (figur \ref{fig:brukere1}). I stedet for å velge modulen <<Servere>> går vi til modul <<System>> som innholder flere undermoduler som kan brukes til oppsett og konfigurasjon av selve den virtuelle maskinen og brukergrensesnittet. 
Vi kan deretter (figur \ref{fig:brukere2}) sette opp en ny gruppe, slette gruppe eller gå videre med å legge til brukere til en gruppe. Vi også se hvilke ressurser som allerede er tildelt en gruppe og eventuelt kan vi fjerne disse rettighetene ved å ta haken et eller flere av alternativene.\\
Dersom vi ønsker å legge til nye brukere i systemet vårt kan vi gjøre det i neste skjermbilde (figur \ref{fig:brukere3}) og på en enkel måte gå videre og legge til eller slette brukere.
Hvis vi ønsker kan vi gå videre til neste skjermbilde og legge til nye ressurser som skal være tilgjengelige for en spesifikk gruppe. Dette kan f.eks. dreie seg om en database, et webområde eller lignende, der alle som er medlem i en gruppe har tilgang.

\section{Hi-fi prototype}
\emph{Følgende avsnitt inneholder ikke figurer av all funksjonalitet i hi-fi prototypen. Det er et flertall skjermbilder som må vises i full bredde. Derfor er samtlige hi-fi mockup flyttet til appendiks \ref{sec:appendiksHiFi} side \pageref{sec:appendiksHiFi}.}

Hi-fi prototypen er en videreutvikling av tidligere prototype som er gjort med hjelp av samme verktøy som skal brukes til ferdig produkt. Ved utvikling av hi-fi prototype ble det brukt  html, css, javascript og php for å sy sammen funksjonalitet og få den til å fungere som en nettside. I figur \ref{fig:hifi_fremside} ser man hvordan fremsiden blir presentert. Dette kan gjerne sammenliknes med den tidligere mockupen figur \ref{fig:lowfi_fremside} side \pageref{fig:lowfi_fremside}. Det er ikke bestemt hva som skal vises på fremsiden enda, men mest sannsynlig blir det meldinger fra lærer og de man jobber i gruppe med.
\begin{figure}[ht]
\includegraphics[width=\textwidth,height=\textheight,keepaspectratio]{./img/prosessdokumentasjon/hifi/fremside.png}
\caption[Hi-fi prototype]{Fremside for EasyDev som hi-fi prototype.}
\label{fig:hifi_fremside}
\end{figure}

Det er en litt annerledes tilnærming til hvordan dialogvinduene til brukeren blir presentert i denne prototypen. Se figur \ref{fig:hifi_brukerdialog}. Dialogen blir presentert lengre ned i forhold til den hovedmenyen som medfører at store deler av det tomme området på fremsiden blir i stedet brukt for visning av dialoger. 

\begin{figure}[ht]
\centering
\includegraphics[width=0.8\textwidth,keepaspectratio,trim = {12cm 6cm 12cm 8cm}, clip]{./img/prosessdokumentasjon/hifi/a2.png}
\caption[Hi-fi brukerdialog]{Eksempel over fremstilling av brukerdialog. Utklippet viser eksempel på ny design hvor dialogen er plassert i forhold til menyen.}
\label{fig:hifi_brukerdialog}
\end{figure}

\section{Kriterier og avgjørelser}
Følgende avsnitt beskriver hvilke og hvorfor vi har tatt de valg som vi har gjort under reisen. Vi skal fokusere på valg kriterier og valg fra et MMI perspektiv men det er også viktig å opplyse av andre kriterier som mer teknisk karakter.

\subsection{Virtuell maskin}
System skal kjøre i en vituell maskin ettersom det er enkelt å både opprette slette eller gjenopprette slik løsning. Da en ny bruker skal starte en maskin trenger denne kun å bli kopiert fra et image og satt opp med riktig brukernavn og passord. Dersom brukeren ødelegger noe i maskinen er det mulig å eksportere konfigurasjonsfiler for valgte moduler og opprette en ny maskin sammen med eksisterende konfigurasjonfiler, til f.eks. Apache.

\subsection{Webbasert}
Ettersom grensesnittet er webbasert blir det tilgjengelig fra hvilken datamaskin som helst uavhengig operativsystem eller nettleser. Systemet vil få en god skalering da det blir enkelt å sette inn eller fjerne moduler. Det vil også være relativ liten terskel for studenter å utvikle egne moduler dersom man synes at noen moduler mangler i systemet. 

\subsection{Bruk av visuelle elementer}
Som tidligere beskrevet i avsnitt \ref{sec:utkast} og \ref{sec:lowfi} har vi valgt å benytte oss av gestalt prinsipper ved plassering av alle visuelle komponenter. Hensikten er at når brukeren for første gang logger på systemet skal man kunne bruke det helt intuitivt og uten å behøve tenkte på hvordan det er implementert.

Et godt eksempel på dette er plassering av alle knapper på fremsiden. Disse har vi valgt å legge horisontalt med samme avstand fra hverandre. Slik plassering danner tilhørighet, man vet at disse henger sammen (\textit{proximity, element connectedness, similarity}). I tillegg forsterkes dette gjennom at vi bruker samme farger på alle de komponentene.\cite{forelesning:tulpesh}. Vi vil gå videre med disse prinsippene ved eventuell videreutvikling av systemet slik at man kunne fargekodet de forskjellige områdene slik at hver område er lett identifiserbart grunnet disse fargekodene. 
Samme prinsipper gjelder også for de undermenyer som åpnes dersom man klikker på noen av de komponentene. Man blir presentert med en liste av muligheter gruppert etter tilhørighet til sin funksjon. 

Toppmenyen er symmetrisk distribuert for å danne en følelse balanse og sentral plassering i layouten. Vi har plassert EasyDev logo i midten og på begge sider av den, komponenter som er kalender og klokke til venstre samt brukerikonet og hurtigvalg til innstillinger til høyre for logoen. 

\subsection{Bruk av farger}
Vi har konsekvent valgt å avgrense oss til grunnpaletten i \textit{FlauUI} grensesnittet. Alle farger som vi kan bruke i løsningen presenteres i form av en matrise i figur \ref{fig:farger}. Dette er $4\times 5$ matrise som kanskje bare har tilgang til tjue farger men gir oss ganske store muligheter. 
\begin{figure}[h]
\begin{tabularx}{\textwidth}{*6{>{\centering\arraybackslash}X}@{}}

\cellcolor{Turquoise} \emph{Turquoise} & \cellcolor{Emerald} \textit{Emerald} & \cellcolor{PeterRiver} \emph{Peter River} & \cellcolor{Amethyst} \emph{Amethyst} & \cellcolor{WetAsphalt} \emph{Wet Asphalt} \\[5ex] 

\cellcolor{GreenSea} \emph{Green Sea} & \cellcolor{Nephritis} \textit{Nephritis} & \cellcolor{BelizeHole} \emph{Belize Hole} & \cellcolor{Wisteria} \emph{Wisteria} & \cellcolor{MidnightBlue} \emph{Midnight Blue} \\[5ex] 

\cellcolor{SunFlower} \emph{Sun Flower} & \cellcolor{Carrot} \textit{Carrot} & \cellcolor{Alizarin} \emph{Alizarin} & \cellcolor{Clouds} \emph{Clouds} & \cellcolor{Concrete} \emph{Concrete} \\[5ex] 
 
\cellcolor{Orange} \emph{Orange} & \cellcolor{Pumpkin} \textit{Pumpkin} & \cellcolor{Pomegranate} \emph{Pomegranate} & \cellcolor{Silver} \emph{Silver} & \cellcolor{Asbestos} \emph{Asbestos} \\[5ex] 

\end{tabularx} 
\caption[Fargekombinasjoner]{Fargekombinasjoner som brukes i hi-fi prototypen}
\label{fig:farger}
\end{figure}
Dersom vi ser på de to øverste og de to nederste radene ser vi at disse bare er forskjellige nyanser av samme farge. Disse har vi valgt å bruke som utheving eller for å danne lav kontrast mellom feks. tabellrader. 
Dersom vi trenger mer kontrast for å varsle brukeren om noe kan vi ta i bruk to \textbf{analoge} farger. Disse er plassert over hverandre i rader 1-3 eller 2-4, f.eks. \emph{Turqoise} og \emph{Sun Flower} eller \emph{Belize Hole} og \emph{Pomegranade}.
Disse gir myke overganger og kan oftest benyttes sammen og gi en myk og fin overgang f.eks. mellom bakgrunnen i et grensesnitt og et element som f.eks. en knapp.
Til sist er det også mulig å benytte seg av to \textbf{komplementærfarger}.
Disse har vi plassert på skrå fra hverandre som \emph{Peter River} og \emph{Orange} eller \emph{Amethyst} og \emph{Sun Flower}. Disse fargene kombinert brukes til å tydelig markere overgangen mellom et objekt og et annet.

Generelt har vi selvsagt ikke benyttet oss av alle disse farger i hi-fi prototypen da mange av dem blir reservert til spesielle tilfeller. Kun mer monotone av farger i figur \ref{fig:farger} blir brukt til komponenter og bakgrunner som ofte forekommer i komponentene.

\subsection{Open Source}
Vi har valgt å basere vår løsning kun på produkter som er tilgjengelige via open source eller som fri kode. Dette er ganske viktig da prosjektet som kunne videreutvikles og ikke skal være avhengig av proprietær teknologi. Videre skal vår kildekode gjøres tilgjengelig slik at det kan gjøres endringer og forbedringer på eksisterende moduler eller lages nye (se avsnitt \ref{sec:hjelpogverkt} for linker). 

\chapter{Brukertesting}
Jeg er helt blank på hva som skal tas med i brukertesting. Her må vi nok lese på noe av teorien og brukete støtteord fra teorien. Noen som har kjøpt boken?
\chapter{Evaluering}
\lettrine[lines=2]{P}{}roduktet vi sitter igjen med etter alt arbeidet er utført er noe annerledes enn det vi så for oss da vi begynte å jobbe med prosjektet. Ambisjonene våre var rettet mot å lage et så godt produkt som mulig, samt å implementere noe funksjonalitet i dybden for å gi brukeren en følelse på hvordan sluttproduktet kunne se ut og hvordan det ville være å jobbe med det. Dette kapittelet vil berøre alle aspektene ved produktet, hvorfor det ble som det ble og hva vil ville ha gjort annerledes.

\section{Produktet}
Resultatet av HiFi-prototypen er det som vi anser som produktet vårt. Den er dog en dårlig pekepinn på hva prosjektet går ut på og dermed vil denne rapporten være det viktigste bidraget til sluttproduktet vårt. Uten rapporten og historien som følger med rundt beslutningene som ble gjort og prosessen underveis vil ikke prototypen ha noen verdi. Prototypen må derfor sees i lys av det man leser i rapporten.


\section{Begrensninger} \label{sec:begrensninger}
Dette punktet har vært den største utfordringen vi har støtt på. I et prosjekt der man skal lage noe ønsker man å har et produkt man kan være fornøyd med, og som vi kunne tenke å jobbe med selv. Dette var jo målet vi hadde satt oss også, men det ble klart ganske tidlig at vi måtte omprioritere, lage ny fremdriftsplan og sette nye mål på hva vi kunne få til innenfor angitt tid.
Den største utfordringen det medførte var i hvor stor grad vi skulle implementere funksjonalitet. Om vi hadde hatt tid kunne vi implementert veldig mye av funksjonaliteten, i alle fall i brukergrensesnittet, men det har altså ikke vært mulig innenfor tidsrammene.
Resultatet har da blitt at vi videreutvikler mockup/prototype fra første innlevering og viser prototypen i "ny drakt" slik at man får en bedre, mer funksjonell prototype, men som ikke har mer funksjonalitet implementert.

Vi har prøvd å forholde oss til prinsipper for MMI-faget. Først og fremst har vi jobbet med farger som passer godt sammen, samt laget en oversiktlig meny som gir brukeren rask tilgang til de forskjellige kategoriene av funksjonalitet vi har ønsket å implementere.
Da produktet vårt er av en teknisk art og ikke ment for den jevne bruker vil ikke alle menypunkter være selvforklarende, men at man har en liten <<læringskurve>> for å bli vant med produktet slik at man kan utnytte all den funksjonalitet produktet tilbyr.
Det skal likevel ikke være behov for å kunne alle disse avanserte funksjonene for at man kunne bruke systemet på en tilfredsstillende måte.


\section{Hva ville vi ha gjort annerledes?}
Premisset for prosjektoppgaven i MMI er annerledes enn det vi la opp til da vi kom opp med vår idé, som var før oppgaven hadde blitt delt ut. Dette har medført at vi ikke fikk tid til å utvikle løsningen vår slik vi ønsket i utgangspunktet, men i vi stedet tilpasset produktet vårt til oppgavens spørsmål og krav.
Det medførte til at vi laget et forslag mer som en konseptløsning med god dokumentasjon som viser ideene og ambisjonene våre i stedet for å lage et best mulig sluttprodukt med mest mulig funksjonalitet. Vi ville likevel stå i fare for å ikke ha tid til å gjøre ferdig produktet eller komme med en god nok løsning som rettferdiggjorde valgene vi hadde tatt.

Videre kunne vi tenkt oss å tatt med en brukerundersøkelse før vi begynte arbeidet for å se hva andre studenter kunne se for seg for løsninger og funksjonalitet. Det ville ikke påvirket arbeidet vårt i nevneverdig grad, men det ville vært interessant å ta med flere av de resultatene inn som tiltenkt funksjonalitet i produktet.

Vi har også diskutert om hvordan man skal gjøre forbedringer i grensesnittet. Det finnes flere alternativer der vi har drøftet muligheten over å bruke litt forskjellige farger på de ulike komponetene. Tanken er at vi kan gruppere de forskjellige kompoenentene gjennom å bruke farger i tilleg til formvariasjoner. På tråd med gestalt prinsippene vil vi da kunne gruppere eller skille forskjellige elementer med hjelp av analoge- eller komplementærfarger eller gruppere med nyasnvariasjoner i fargene. Dersom vi hadde litt mer tid til hands kunde vi implemetert slik funksjonalitet i tilleg til den eksisterende og gjennomført en ekstra brukertesting på de. Slik brukeretesting borde da bli gjennomført på mest sannsynnerlig brukergrupper og tolke resultatene mot en hypotese for å statistik finne ut hvilken av disse som brukere foretrekker eller borde implementeres i produktet.

\section{Veien videre}
EasyDev føddes som en spontan idé for gruppeprosjekt i MMI faget. Dersom vi ser det frå vårt studentperspekitv er dette en god idé som vi ønsker å bygge videre på. Dette er et produkt som ikke finnes i markedet i den form som er spesifisert i oppgaven. Det er fult mulig å forfine produktet og lansere det utenfor skoleverden. 
Her drøfter vi muligheter til en komersiell lasnering i f.eks. \textit{skyen}. Det vil dog forutsette flere tilleg og eventuelt en modul med fullverdi dektop løsning i skyen. Desktop i \textit{skyen} er ikke noen nytt og finnes tilgjengelig fra før. Idag ser vi oftest tre trender i <<\textit{clound computing}>> der vi kan antigen har tilgang til \textit{SaaS - software as service}, \textit{PaaS - platform as service} eller \textit{DaaS - desktop as service}. 
Det er ingen eller få muligheter å få både et operativsystem, desktop og server som service. 
I vår løsning tilbyr man i tillegg alle disse verktøy og funksjonalitet som beskrives i appendiks \ref{app:funksjonalitet} (side \pageref{app:funksjonalitet}). Noe som vil gi en meget unik kombinasjon dersom man legger til dektop funksjonalitet.
Veien videre borde begynne med at man gjennomfører en grunnlig markedsundersøkelse med hensikt å finne dersom slik etterspørsel finnes i det globale markedet. Deretter må kartlegges produktets styrker, svakheter, muligheter og trusler for å finne ut produktets konkurransestyrke og potensial.

%Appendix for vedlegg og andre ting som ikke direkte passer inn i resten av rapporten

\bibliographystyle{plain}
\addcontentsline{toc}{chapter}{Bibliografi}
\bibliography{bibliografi}
\addcontentsline{toc}{chapter}{Vedlegg}
\appendix

\chapter{Her kommer appendix}
Ikke sikker at vi skal ha appendix da vi skal legge inn nesten alle figurer rett i teksten men filen kan være her foreløpig.

% IKKE SLETT VIL HA DENNE SOM EKSEMPEL
%\begin{figure}[ht]
% \includegraphics[angle=90 ,width=\textwidth,height=\textheight,keepaspectratio]{./img/appendix/diagram/klassestruktur_uml.png}
% \caption{Innledende UML diagram. brukt for generering av grunnleggende klasser.}
% %Her kommer en kabel for kryssreferering i teksten til figuren
% \label{fig:uml_diag}
%\end{figure}

\end{document}